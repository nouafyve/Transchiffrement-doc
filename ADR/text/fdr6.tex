\begin{center}
\begin{tabular}{|>{\columncolor[gray]{.8}}m{8cm}|m{8cm}|}
\hline
 Intitulé du risque &  Perte des données du projet \\
\hline
 Description du risque & Une mauvaise manipulation ou un crash du serveur où sont stockées les données résultent en une perte de toute les données. \\
\hline
Pilote du risque & Yves NOUAFO \\
\hline
\end{tabular}
\end{center}

\begin{center}
\begin{tabular}{|>{\columncolor[gray]{.8}}m{3.8cm}|m{3.8cm}|>{\columncolor[gray]{.8}}m{3.8cm}|m{3.8cm}|}
\hline
Indice de gravité & 4 &Date de début d'effet& Début de la conception \\
\hline
Indice de probabilité & 1 & Date de fin d'effet & Fin du projet\\
\hline
Criticité \footnotemark[1] & À surveiller &  & \\
\hline
État \footnotemark[2] & \hl{Inactif} & Catégorie \footnotemark[3] & Technique\\
\hline
\end{tabular}
\end{center}

\begin{center}
\begin{tabular}{|m{5cm}|m{11cm}|}
\hline
\rowcolor[gray]{.8} Actions préventives & Description\\
\hline
Sauvegarde & Tous les soirs, une copie des données sera faite en local sur chaque ordinateur, ce qui permettra de recréer le projet à partir de ces données \\
\hline
\end{tabular}
\end{center}

\begin{center}
\begin{tabular}{|m{5cm}|m{11cm}|}
\hline
\rowcolor[gray]{.8} Actions correctives & Description\\
\hline
%\aRemplir{Nom de l'action} & \aRemplir{Lister les actions corrective en étant le plus précis possible} \\
Récupération des données & Utilisation de la sauvegarde locale sur un des ordinateurs du groupe \\
\hline
\end{tabular}
\end{center}




\footnotetext[1]{Acceptable / À surveiller / Critique}
\footnotetext[2]{Actif / Inactif}
\footnotetext[3]{Compétences / Planning / Technique}
