\begin{center}
\begin{tabular}{|>{\columncolor[gray]{.8}}m{8cm}|m{8cm}|}
\hline
 Intitulé du risque & Ne pas livrer le client à temps \\
\hline
 Description du risque & Des retards sur le projet empêchent de livrer les lots selon le planning prévu. \\
\hline
Pilote du risque & Émile GÉNÉRAT \\
\hline
\end{tabular}
\end{center}

\begin{center}
\begin{tabular}{|>{\columncolor[gray]{.8}}m{3.8cm}|m{3.8cm}|>{\columncolor[gray]{.8}}m{3.8cm}|m{3.8cm}|}
\hline
Indice de gravité & 3 &Date de début d'effet& Début de la conception \\
\hline
Indice de probabilité & 1 & Date de fin d'effet & Fin du projet\\
\hline
Criticité \footnotemark[1] & Acceptable &  & \\
\hline
État \footnotemark[2] & Actif & Catégorie \footnotemark[3] & Planning\\
\hline
\end{tabular}
\end{center}

\begin{center}
\begin{tabular}{|m{5cm}|m{11cm}|}
\hline
\rowcolor[gray]{.8} Actions préventives & Description\\
\hline
 Réunions quotidiennes & Pour ne pas déborder du planning, cette réunion nous permettra de présenter l'avancement de la journée, et ce qui a été réalisé. \\
\hline
\end{tabular}
\end{center}

\begin{center}
\begin{tabular}{|m{5cm}|m{11cm}|}
\hline
\rowcolor[gray]{.8} Actions correctives & Description\\
\hline
Redéfinir le pérmètre du projet & Si le temps n'est pas suffisant pour terminer le projet, nous devrons nous concerter avec le client pour redéfinir les fonctionnalités minimales attendues.\\
\hline
\end{tabular}
\end{center}



\footnotetext[1]{Acceptable / À surveiller / Critique}
\footnotetext[2]{Actif / Inactif}
\footnotetext[3]{Compétences / Planning / Technique}

