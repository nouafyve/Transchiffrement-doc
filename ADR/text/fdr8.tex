\begin{center}
\begin{tabular}{|>{\columncolor[gray]{.8}}m{8cm}|m{8cm}|}
\hline
 Intitulé du risque &  Modification du cahier des charges \\
\hline
 Description du risque & Le client souhaite modifier ou ajouter des fonctionnalités au cahier des charges  \\
\hline
Pilote du risque & Julien BOURDON \\
\hline
\end{tabular}
\end{center}

\begin{center}
\begin{tabular}{|>{\columncolor[gray]{.8}}m{3.8cm}|m{3.8cm}|>{\columncolor[gray]{.8}}m{3.8cm}|m{3.8cm}|}
\hline
Indice de gravité & 2 &Date de début d'effet& Début de la conception \\
\hline
Indice de probabilité & 2 & Date de fin d'effet & Fin du projet\\
\hline
Criticité \footnotemark[1] & Acceptable &  & \\
\hline
État \footnotemark[2] & Actif & Catégorie \footnotemark[3] & Technique\\
\hline
\end{tabular}
\end{center}

\begin{center}
\begin{tabular}{|m{5cm}|m{11cm}|}
\hline
\rowcolor[gray]{.8} Actions préventives & Description\\
\hline
 Réunions hebdomadaires avec le client & Pour éviter de créer de nouveaux besoins, ces réunions avec le client nous aideront à respecter ses attentes et à limiter les modifications \\
\hline
\end{tabular}
\end{center}

\begin{center}
\begin{tabular}{|m{5cm}|m{11cm}|}
\hline
\rowcolor[gray]{.8} Actions correctives & Description\\
\hline
%\aRemplir{Nom de l'action} & \aRemplir{Lister les actions corrective en étant le plus précis possible} \\
Prise en compte des modifications & Une réorganisation du planning sera faite si nous avons le temps et les moyens de réaliser cette modification, sinon cette modification sera considérée comme optionnelle en accord avec le client. \\
\hline
\end{tabular}
\end{center}


\footnotetext[1]{Acceptable / À surveiller / Critique}
\footnotetext[2]{Actif / Inactif}
\footnotetext[3]{Compétences / Planning / Technique}
