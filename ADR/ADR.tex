\documentclass[a4paper,11pt,french]{article}
\usepackage[utf8]{inputenc}

\usepackage[T1]{fontenc}
\usepackage[francais]{babel} 
\usepackage[top=2cm, bottom=2cm, left=2cm, right=2cm, includeheadfoot]{geometry} %pour les marges
\usepackage{lmodern}
\usepackage{pict2e}

\usepackage{fancyhdr} % Required for custom headers
\usepackage{lastpage} % Required to determine the last page for the footer
\usepackage{extramarks} % Required for headers and footers
\usepackage{graphicx} % Required to insert images
\usepackage{tabularx, longtable}
\usepackage{color, colortbl}
\usepackage{lscape}
%\usepackage[hidelinks]{hyperref}
\usepackage{longtable}
\usepackage{multirow}
\usepackage{rotating}
%\usepackage{pgfgantt}
%\usepackage{pgfcalendar}
%\usepackage{ifthen}
\usepackage{gensymb}
\usepackage{soulutf8}

\linespread{1.1} % Line spacing

% Set up the header and footer
\pagestyle{fancy}
\lhead{\textbf{\hmwkClass -- \hmwkSubject \\ \hmwkTitle \\ \hmwkDocName}} % Top left header
\rhead{\includegraphics[width=10em]{../images/logo_univ.png}}
\lfoot{\lastxmark} % Bottom left footer
\cfoot{} % Bottom center footer
\rfoot{Page\ \thepage\ / \pageref{LastPage}} % Bottom right footer
\renewcommand\headrulewidth{0.4pt} % Size of the header rule
\renewcommand\footrulewidth{0.4pt} % Size of the footer rule

\setlength{\headheight}{40pt}

\newcommand{\hmwkTitle}{Transchiffrement} % Assignment title
\newcommand{\hmwkClass}{Master 2 SSI } % Course/class
\newcommand{\hmwkAuthorName}{Emile GÉNÉRAT} % Your name
\newcommand{\hmwkSubject}{Conduite de projet} % Subject
\newcommand{\hmwkDocName}{Analyse des Risques} % Document name

\newcommand{\version}{1.2} % Document version
\newcommand{\docDate}{29 Janvier 2014} % Document date
\newcommand{\checked}{Julien BOURDON} % Checker name
\newcommand{\approved}{} % Approver name

\makeatletter
\newcommand{\resettranslate}{\let\translate\@firstofone}
\makeatother

\definecolor{gris}{rgb}{0.95, 0.95, 0.95}

\title{
\vspace{2in}
\textmd{\textbf{\hmwkClass :\ \hmwkTitle}}\\
\normalsize\vspace{0.1in}\small{Due\ on\ \hmwkDueDate}\\
\vspace{0.1in}\large{\textit{\hmwkClassInstructor\ \hmwkClassTime}}
\vspace{3in}
}

\author{\hmwkAuthorName}
\date{} % Insert date here if you want it to appear below your name


\usepackage{amsmath}
\begin{document}
\newcount\startdate
\newcount\daynum
%\pgfcalendardatetojulian{2013-01-021}{\startdate}
\pagestyle{fancy}

\vspace*{5cm}
\begin{center}\textbf{\Huge{\hmwkDocName}}\end{center}
\vspace*{4.5cm}
	

\fcolorbox{black}{gris}{
\begin{minipage}{15cm}
\begin{tabularx}{10cm}{lXl}
	\bfseries{Version} & & \version\\
	& & \\
	\bfseries{Date} & & \docDate\\
	& & \\
	\bfseries{Rédigé par} & & \hmwkAuthorName \\
	& & \\
	\bfseries{Relu par} & & \checked \\
	& & \\
	\bfseries{Approuvé par} & & \approved \\
	& & \\
\end{tabularx}
\end{minipage}
}

\newpage

%Tableau de mises à jour
\vspace*{1cm}
\begin{center}
\textbf{\huge{Versions}}\\
\vspace*{3cm}
	\begin{tabularx}{16cm}{|c|c|X|}
	\hline
	\bfseries{Version} & \bfseries{Date} & \bfseries{Modifications réalisées}\\
	\hline
	1.0 & 13/12/2013 & Création\\
	\hline
	1.1 & 22/01/2014 & Modification pour prise en compte des remarques du Client et de l'audit \\
	\hline
	1.2 & 27/01/2014 & \hl{Modification pour prise en compte des remarques de l'audit du 24 janvier, cf CR\_A\_2013-01-24.pdf} \\
	\hline
	\end{tabularx}
\end{center}

%La table des matières
\clearpage
\tableofcontents
\clearpage


\newpage
\section{Documents de référence}
\begin{itemize}
\item Plan de Développement v1.1
\end{itemize}


\newpage
\section{Portefeuille de risques}

\subsection{Évaluation des risques}

\hl{Les différents risques que nous avons découverts sont évalués en fonction leur probabilité d'apparition, et de la gravité de leur impact si ils se produisent, sur la satisfaction du client.}

\begin{table}[!h]
\begin{tabular}{|>{\centering\arraybackslash}m{4.5cm}|l|}
\hline
\multirow{4}*{Indice de probabilité} & 1 : très peu probable (P < 5\%) \\
 & 2 : peu probable  (5\% < P < 20\%) \\
 & 3 : probable  (20\% < P < 50\%) \\
 & 4 : très probable  (P > 50\%)\\
 \hline
\multirow{4}*{Indice de gravité} & 1 : faible conséquence sur l'appréciation du client \\
 & 2 : conséquence significative sur l'appréciation du client \\
 & 3 : conséquence importante sur l'appréciation du client \\
 & 4 : mécontentement assuré du client \\
\hline
\end{tabular}
\end{table}

\hl{Nous devons porter une attention particulière aux risques \textit{critiques}, et dans une moindre mesure ceux \textit{à surveiller}.

Si il n'y a pas de changement majeur, nous actualisons l'évaluation des risques chaque semaine.}


\subsection{Rôle du pilote de risque}

\hl{Lorsqu'un risque devient actif, le pilote de risque doit s'assurer que les mesures de corrections sont :}
\begin{itemize}
\item \hl{correctement appliquées ;}
\item \hl{suffisantes.}
\end{itemize}
\hl{Si ce n'est pas le cas, il doit proposer de nouvelles mesures en accord avec les acteurs concernés.

Lors de la mise en place d'un plan de contournement (des actions correctives), le pilote de risque s'assure avec le chef de projet que le client a une bonne visibilité sur les actions prises.}

\subsection*{Portefeuille de risques}
\begin{table}[!h]
\begin{tabular}{|c|c|c|c|c|}
\hline
Probabilité/Gravité & 1 & 2 & 3 & 4 \\
\hline
1 & \cellcolor{green} acceptable & \cellcolor{green} acceptable & \cellcolor{green} acceptable & \cellcolor{yellow} à surveiller \\
\hline
2 & \cellcolor{green} acceptable & \cellcolor{green} acceptable & \cellcolor{yellow} à surveiller & \cellcolor{red} critique \\
\hline
3 & \cellcolor{green} acceptable & \cellcolor{yellow} à surveiller & \cellcolor{red} critique & \cellcolor{red} critique \\
\hline
4 & \cellcolor{green} acceptable & \cellcolor{red} critique & \cellcolor{red} critique & \cellcolor{red} critique\\
\hline
\end{tabular}
\end{table}

\begin{landscape}

\begin{longtable}{|m{1.5cm}|m{4cm}|m{3cm}|m{3.2cm}|m{2.2cm}|m{2cm}|m{2cm}|m{2.2cm}|}
\hline
\rowcolor[gray]{.8}
Numéro & Risque & Date de création & Pilote & Catégorie \footnotemark[1] & Indice de probabilité & Indice de gravité & Criticité \footnotemark[2] \\ 
\hline
\endfirsthead
\hline
\rowcolor[gray]{.8}
Numéro & Risque & Date de création & Pilote & Catégorie \footnotemark[1] & Indice de probabilité & Indice d'impact & Criticité \footnotemark[2] \\ 
\hline

\endhead
n01 & Ne pas livrer le client à temps & 13/12/2013 & Émile GÉNÉRAT & Planning & 1 & 3 & Acceptable \\
\hline
n02 & Insatisfaction du client & 13/12/2013 & Émile GÉNÉRAT & Planning & \hl{2} & 4 &  \hl{Critique}\\
\hline
n03 & Retard, algorithme prêt trop tard & 13/12/2013 & Émile GÉNÉRAT & Planning & 3 & 2 & À surveiller \\
\hline
n04 & Niveau technique insuffisant & 13/12/2013 & Jean-Baptiste SOUCHAL &  Compétence & 1 & 2 & Acceptable \\
\hline
n05 & Implication insuffisante, ou absences & 13/12/2013 & Julien BOURDON & Planning & 1 & 3 & Acceptable \\
\hline
n06 & Perte des données du projet & 13/12/2013 & Yves NOUAFO & Technique & 1 & 4 & À surveiller \\
\hline
n07 & Performances insuffisantes (transchiffrement trop long) & 13/12/2013 & Ouissem HAMDANI & Technique & 1 & 3 & Acceptable \\
\hline
n08 & Modification du Cahier des Charges & 13/12/2013 & Julien BOURDON & Technique & \hl{3} & \hl{3} & \hl{Critique} \\
\hline

\end{longtable}
\footnotetext[1]{Compétences / Planning / Technique}
\footnotetext[2]{Acceptable / À surveiller / Critique}

\end{landscape}

\newpage

\subsection{Fiches de risque}

\subsubsection{Risque n \no 1 : Ne pas livrer le client à temps}
\begin{center}
\begin{tabular}{|>{\columncolor[gray]{.8}}m{8cm}|m{8cm}|}
\hline
 Intitulé du risque & Ne pas livrer le client à temps \\
\hline
 Description du risque & Des retards sur le projet empêchent de livrer les lots selon le planning prévu. \\
\hline
Pilote du risque & Émile GÉNÉRAT \\
\hline
\end{tabular}
\end{center}

\begin{center}
\begin{tabular}{|>{\columncolor[gray]{.8}}m{3.8cm}|m{3.8cm}|>{\columncolor[gray]{.8}}m{3.8cm}|m{3.8cm}|}
\hline
Indice de gravité & 3 &Date de début d'effet& Début de la conception \\
\hline
Indice de probabilité & 1 & Date de fin d'effet & Fin du projet\\
\hline
Criticité \footnotemark[1] & Acceptable &  & \\
\hline
État \footnotemark[2] & \hl{Inactif} & Catégorie \footnotemark[3] & Planning\\
\hline
\end{tabular}
\end{center}

\begin{center}
\begin{tabular}{|m{5cm}|m{11cm}|}
\hline
\rowcolor[gray]{.8} Actions préventives & Description\\
\hline
 Réunions quotidiennes & Pour ne pas déborder du planning, cette réunion nous permettra de présenter l'avancement de la journée, et ce qui a été réalisé. \\
\hline
\end{tabular}
\end{center}

\begin{center}
\begin{tabular}{|m{5cm}|m{11cm}|}
\hline
\rowcolor[gray]{.8} Actions correctives & Description\\
\hline
Redéfinir le pérmètre du projet & Si le temps n'est pas suffisant pour terminer le projet, nous devrons nous concerter avec le client pour redéfinir les fonctionnalités minimales attendues.\\
\hline
\end{tabular}
\end{center}



\footnotetext[1]{Acceptable / À surveiller / Critique}
\footnotetext[2]{Actif / Inactif}
\footnotetext[3]{Compétences / Planning / Technique}



\newpage
\subsubsection{Risque \no  2 : Insatisfaction du client}
\begin{center}
\begin{tabular}{|>{\columncolor[gray]{.8}}m{8cm}|m{8cm}|}
\hline
 Intitulé du risque &  Insatisfaction du client\\
\hline
 Description du risque & Malgré la finition du projet, le rendu n'est pas conforme aux attentes du client. \\
\hline
Pilote du risque &  Émile GÉNÉRAT \\
\hline
\end{tabular}
\end{center}

\begin{center}
\begin{tabular}{|>{\columncolor[gray]{.8}}m{3.8cm}|m{3.8cm}|>{\columncolor[gray]{.8}}m{3.8cm}|m{3.8cm}|}
\hline
Indice de gravité & 4 &Date de début d'effet& Début de la conception \\
\hline
Indice de probabilité & \hl{2} & Date de fin d'effet & Fin du projet\\
\hline
Criticité \footnotemark[1] & \hl{Critique}  &  & \\
\hline
État \footnotemark[2] & Actif & Catégorie \footnotemark[3] & Planning\\
\hline
\end{tabular}
\end{center}

\begin{center}
\begin{tabular}{|m{5cm}|m{11cm}|}
\hline
\rowcolor[gray]{.8} Actions préventives & Description\\
\hline
Réunions hebdomadaires avec le client & Pour éviter de partir dans une mauvaise direction, ces réunions avec le client nous aideront à respecter ses attentes.\\
\hline
Démonstration et mises à disposition de code & Le planning prévisionnel doit contenir des dates de démonstration. À ces échéances, nous présenterons de nouvelles fonctionnalités, pour avoir régulièrement des retours sur notre travail.\\
\hline
\end{tabular}
\end{center}



\begin{center}
\begin{tabular}{|m{5cm}|m{11cm}|}
\hline
\rowcolor[gray]{.8} Actions correctives & Description\\
\hline
\hl{Suivi des remarques du client} & \hl{Lorsque le client émet des remarques, il faut :}
\begin{itemize}
\item \hl{Les tracer;}
\item \hl{Attribuer la tâche à un réalisateur, avec une charge et une échéance;}
\item \hl{Communiquer avec le client, pour lui présenter les corrections réalisées, et obtenir son avis.}
\end{itemize} \\
\hline
Correction de la livraison & Prendre en compte les remarques du client pendant la période de correction. \\
\hline
\end{tabular}
\end{center}



\footnotetext[1]{Acceptable / À surveiller / Critique}
\footnotetext[2]{Actif / Inactif}
\footnotetext[3]{Compétences / Planning / Technique}


\newpage
\subsubsection{Risque \no  3 : Retard, algorithme prêt trop tard}
\begin{center}
\begin{tabular}{|>{\columncolor[gray]{.8}}m{8cm}|m{8cm}|}
\hline
 Intitulé du risque &  Retard, algorithme prêt trop tard\\
\hline
 Description du risque & L'algorithme de recherche est prêt trop tard, et le temps d'exécution est trop court pour produire des résultats.  \\
\hline
Pilote du risque & Émile GÉNÉRAT \\
\hline
\end{tabular}
\end{center}

\begin{center}
\begin{tabular}{|>{\columncolor[gray]{.8}}m{3.8cm}|m{3.8cm}|>{\columncolor[gray]{.8}}m{3.8cm}|m{3.8cm}|}
\hline
Indice de gravité & 2 &Date de début d'effet& Début de la conception \\
\hline
Indice de probabilité & 3 & Date de fin d'effet & Fin du projet\\
\hline
Criticité \footnotemark[1] & À surveiller & & \\
\hline
État \footnotemark[2] & Actif & Catégorie \footnotemark[3] & Planning\\
\hline
\end{tabular}
\end{center}

\begin{center}
\begin{tabular}{|m{5cm}|m{11cm}|}
\hline
\rowcolor[gray]{.8} Actions préventives & Description\\
\hline
Réunions hebdomadaires avec le client & Demander des indications à notre Client. \\
\hline
Recherche & Travail de compréhension de l'attaque sur MD5 à partir d'articles publiés et récupération d'aide sur l'algorithme utilisé dans cette recherche. \\
\hline
\end{tabular}
\end{center}

\begin{center}
\begin{tabular}{|m{5cm}|m{11cm}|}
\hline
\rowcolor[gray]{.8} Actions correctives & Description\\
\hline
Étude rédigée & L'étude rédigée permet de garder une trace des différentes recherches qui ont été faites pour mettre au point l'algorithme. \\
\hline
\end{tabular}
\end{center}




\footnotetext[1]{Acceptable / À surveiller / Critique}
\footnotetext[2]{Actif / Inactif}
\footnotetext[3]{Compétences / Planning / Technique}


\newpage
\subsubsection{Risque \no  4 : Niveau technique insuffisant}
\begin{center}
\begin{tabular}{|>{\columncolor[gray]{.8}}m{8cm}|m{8cm}|}
\hline
 Intitulé du risque &  Niveau technique insuffisant\\
\hline
 Description du risque & Un manque de technique d'un ou plusieurs des étudiants nous handicape et ne nous permet pas de réaliser le projet en totalité  \\
\hline
Pilote du risque & Jean-Baptiste SOUCHAL \\
\hline
\end{tabular}
\end{center}

\begin{center}
\begin{tabular}{|>{\columncolor[gray]{.8}}m{3.8cm}|m{3.8cm}|>{\columncolor[gray]{.8}}m{3.8cm}|m{3.8cm}|}
\hline
Indice de gravité & 2 &Date de début d'effet& Début de la conception \\
\hline
Indice de probabilité & 1 & Date de fin d'effet & Fin du projet\\
\hline
Criticité \footnotemark[1] & Acceptable &  & \\
\hline
État \footnotemark[2] & \hl{Inactif} & Catégorie \footnotemark[3] & Compétence\\
\hline
\end{tabular}
\end{center}

\begin{center}
\begin{tabular}{|m{5cm}|m{11cm}|}
\hline
\rowcolor[gray]{.8} Actions préventives & Description\\
\hline
Formation & Que ce soit par le biais de tutoriels ou d'une explication d'un autre étudiant, l'apprentissage et l'approfondissement des connaissances devra être fait en amont des tâches.\\
\hline
\end{tabular}
\end{center}

\begin{center}
\begin{tabular}{|m{5cm}|m{11cm}|}
\hline
\rowcolor[gray]{.8} Actions correctives & Description\\
\hline
%\aRemplir{Nom de l'action} & \aRemplir{Lister les actions corrective en étant le plus précis possible} \\
Entre-aide & Si un étudiant n'est pas capable de développer sa partie à cause d'un manque de connaissance, il sera aidé par un autre étudiant sur l'ensemble de sa tâche ce qui lui fera comprendre comment il faut faire. \\
\hline
\end{tabular}
\end{center}




\footnotetext[1]{Acceptable / À surveiller / Critique}
\footnotetext[2]{Actif / Inactif}
\footnotetext[3]{Compétences / Planning / Technique}


\newpage
\subsubsection{Risque \no  5 : Implication insuffisante ou absences}
\begin{center}
\begin{tabular}{|>{\columncolor[gray]{.8}}m{8cm}|m{8cm}|}
\hline
 Intitulé du risque &  Implication insuffisante ou absences \\
\hline
 Description du risque & Un ou plusieurs étudiants n'accomplissent pas les tâches qui leur ont été données ou sont absents trop longtemps pour ne pas avoir le temps de les faire. \\
\hline
Pilote du risque & Julien BOURDON \\
\hline
\end{tabular}
\end{center}

\begin{center}
\begin{tabular}{|>{\columncolor[gray]{.8}}m{3.8cm}|m{3.8cm}|>{\columncolor[gray]{.8}}m{3.8cm}|m{3.8cm}|}
\hline
Indice de gravité & 3 &Date de début d'effet& Début de la conception \\
\hline
Indice de probabilité & 1 & Date de fin d'effet & Fin du projet\\
\hline
Criticité \footnotemark[1] & Acceptable &  & \\
\hline
État \footnotemark[2] & Actif & Catégorie \footnotemark[3] & Planning\\
\hline
\end{tabular}
\end{center}

\begin{center}
\begin{tabular}{|m{5cm}|m{11cm}|}
\hline
\rowcolor[gray]{.8} Actions préventives & Description\\
\hline
Réunions quotidiennes & Pour surveiller que chacun accomplit bien les tâches qui lui sont affectées, une réunion quotidienne permettra de sentir la motivation des étudiants.\\
\hline
Prévoir les absences & Pour que les absences ne soient pas dérangeantes pour le projet (absences des deux personnes qui travaillent sur le même point), les absences prévues sont publiées à l'avance. \\
\hline
\end{tabular}
\end{center}

\begin{center}
\begin{tabular}{|m{5cm}|m{11cm}|}
\hline
\rowcolor[gray]{.8} Actions correctives & Description\\
\hline
%\aRemplir{Nom de l'action} & \aRemplir{Lister les actions corrective en étant le plus précis possible} \\
Réaffectation des tâches & Un étudiant qui peine à travailler sur une tâche car elle ne l'intéresse pas pourra se voir affecter à une autre tâche pour tenter de le remotiver (cas exceptionnel).\\
\hline
Sanction & Un étudiant qui est trop absent sera amené à s'expliquer avec le professeur encadrant.\\
\hline
\end{tabular}
\end{center}




\footnotetext[1]{Acceptable / À surveiller / Critique}
\footnotetext[2]{Actif / Inactif}
\footnotetext[3]{Compétences / Planning / Technique}


\newpage
\subsubsection{Risque \no  6 : Perte des données du projet }
\begin{center}
\begin{tabular}{|>{\columncolor[gray]{.8}}m{8cm}|m{8cm}|}
\hline
 Intitulé du risque &  Perte des données du projet \\
\hline
 Description du risque & Une mauvaise manipulation ou un crash du serveur où sont stockées les données résultent en une perte de toute les données. \\
\hline
Pilote du risque & Yves NOUAFO \\
\hline
\end{tabular}
\end{center}

\begin{center}
\begin{tabular}{|>{\columncolor[gray]{.8}}m{3.8cm}|m{3.8cm}|>{\columncolor[gray]{.8}}m{3.8cm}|m{3.8cm}|}
\hline
Indice de gravité & 4 &Date de début d'effet& Début de la conception \\
\hline
Indice de probabilité & 1 & Date de fin d'effet & Fin du projet\\
\hline
Criticité \footnotemark[1] & À surveiller &  & \\
\hline
État \footnotemark[2] & Actif & Catégorie \footnotemark[3] & Technique\\
\hline
\end{tabular}
\end{center}

\begin{center}
\begin{tabular}{|m{5cm}|m{11cm}|}
\hline
\rowcolor[gray]{.8} Actions préventives & Description\\
\hline
Sauvegarde & Tous les soirs, une copie des données sera faite en local sur chaque ordinateur, ce qui permettra de recréer le projet à partir de ces données \\
\hline
\end{tabular}
\end{center}

\begin{center}
\begin{tabular}{|m{5cm}|m{11cm}|}
\hline
\rowcolor[gray]{.8} Actions correctives & Description\\
\hline
%\aRemplir{Nom de l'action} & \aRemplir{Lister les actions corrective en étant le plus précis possible} \\
Récupération des données & Utilisation de la sauvegarde locale sur un des ordinateurs du groupe \\
\hline
\end{tabular}
\end{center}




\footnotetext[1]{Acceptable / À surveiller / Critique}
\footnotetext[2]{Actif / Inactif}
\footnotetext[3]{Compétences / Planning / Technique}


\newpage
\subsubsection{Risque \no  7 : Performances insuffisantes }
\begin{center}
\begin{tabular}{|>{\columncolor[gray]{.8}}m{8cm}|m{8cm}|}
\hline
 Intitulé du risque &  Performances insuffisantes \\
\hline
 Description du risque & Une mauvaise implémentation du code ne satisfait pas les contraintes de rapidité pour rendre le transchiffrement du proxy transparent. \\
\hline
Pilote du risque & Ouissem HAMDANI \\
\hline
\end{tabular}
\end{center}

\begin{center}
\begin{tabular}{|>{\columncolor[gray]{.8}}m{3.8cm}|m{3.8cm}|>{\columncolor[gray]{.8}}m{3.8cm}|m{3.8cm}|}
\hline
Indice de gravité & 3 &Date de début d'effet& Début de la conception \\
\hline
Indice de probabilité & 1 & Date de fin d'effet & Fin du projet\\
\hline
Criticité \footnotemark[1] & Acceptable &  & \\
\hline
État \footnotemark[2] & \hl{Inactif} & Catégorie \footnotemark[3] & Technique\\
\hline
\end{tabular}
\end{center}

\begin{center}
\begin{tabular}{|m{5cm}|m{11cm}|}
\hline
\rowcolor[gray]{.8} Actions préventives & Description\\
\hline
Tests d'efficacité & Plusieurs tests seront faits pour que les temps ne dépassent pas ceux spécifiés dans la STB. \\
\hline
\end{tabular}
\end{center}

\begin{center}
\begin{tabular}{|m{5cm}|m{11cm}|}
\hline
\rowcolor[gray]{.8} Actions correctives & Description\\
\hline
%\aRemplir{Nom de l'action} & \aRemplir{Lister les actions corrective en étant le plus précis possible} \\
Optimisation des algorithmes & Nous profilerons l'application, pour chercher les tâches qui consomment le plus de temps. Nous pourrons modifier nos algorithmes, et/ou utiliser de nouveaux algorithmes pour le chiffrement.  \\
\hline
\end{tabular}
\end{center}




\footnotetext[1]{Acceptable / À surveiller / Critique}
\footnotetext[2]{Actif / Inactif}
\footnotetext[3]{Compétences / Planning / Technique}


\newpage
\subsubsection{Risque \no  8 : Modification du cahier des charges}
\begin{center}
\begin{tabular}{|>{\columncolor[gray]{.8}}m{8cm}|m{8cm}|}
\hline
 Intitulé du risque &  Modification du cahier des charges \\
\hline
 Description du risque & Le client souhaite modifier ou ajouter des fonctionnalités au cahier des charges  \\
\hline
Pilote du risque & Julien BOURDON \\
\hline
\end{tabular}
\end{center}

\begin{center}
\begin{tabular}{|>{\columncolor[gray]{.8}}m{3.8cm}|m{3.8cm}|>{\columncolor[gray]{.8}}m{3.8cm}|m{3.8cm}|}
\hline
Indice de gravité & \hl{3} &Date de début d'effet& Début de la conception \\
\hline
Indice de probabilité & \hl{3} & Date de fin d'effet & Fin du projet\\
\hline
Criticité \footnotemark[1] & \hl{Critique} &  & \\
\hline
État \footnotemark[2] & Actif & Catégorie \footnotemark[3] & Technique\\
\hline
\end{tabular}
\end{center}

\begin{center}
\begin{tabular}{|m{5cm}|m{11cm}|}
\hline
\rowcolor[gray]{.8} Actions préventives & Description\\
\hline
 Réunions hebdomadaires avec le client & Pour éviter de créer de nouveaux besoins, ces réunions avec le client nous aideront à respecter ses attentes et à limiter les modifications \\
\hline
\end{tabular}
\end{center}

\begin{center}
\begin{tabular}{|m{5cm}|m{11cm}|}
\hline
\rowcolor[gray]{.8} Actions correctives & Description\\
\hline
%\aRemplir{Nom de l'action} & \aRemplir{Lister les actions corrective en étant le plus précis possible} \\
Prise en compte des modifications & 
\hl{Lorsque le client demande une modification du cahier des charges, les actions à suivre sont les suivantes :}
\begin{itemize}
\item \hl{Évaluer la charge des modifications demandées.}
\item \hl{Communiquer cette information au client.}
\item \hl{Si la charge totale restante est supérieure aux ressources disponibles, alors il faut redéfinir les tâches prioritaires avec le client.}
\item \hl{Proposer un nouveau planning prévisionnel.}
\item \hl{Demander une confirmation au client.}
\end{itemize}\\


\hline
\end{tabular}
\end{center}


\footnotetext[1]{Acceptable / À surveiller / Critique}
\footnotetext[2]{Actif / Inactif}
\footnotetext[3]{Compétences / Planning / Technique}


\end{document}