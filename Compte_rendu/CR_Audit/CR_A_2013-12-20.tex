\documentclass[a4paper,11pt,french]{article}
\usepackage[utf8]{inputenc}

\usepackage[T1]{fontenc}
\usepackage[francais]{babel} 
\usepackage[top=2cm, bottom=2cm, left=2cm, right=2cm, includeheadfoot]{geometry} %pour les marges
\usepackage{lmodern}
\usepackage{pict2e}
\usepackage{fancyhdr} % Required for custom headers
\usepackage{lastpage} % Required to determine the last page for the footer
\usepackage{extramarks} % Required for headers and footers
\usepackage{graphicx} % Required to insert images
\usepackage{tabularx, longtable}
\usepackage{color, colortbl}
\usepackage{lscape}
%\usepackage[hidelinks]{hyperref}
\usepackage{longtable}
\usepackage{multirow}
\usepackage{rotating}
%\usepackage{pgfgantt}
%\usepackage{pgfcalendar}
%\usepackage{ifthen}
\usepackage{gensymb}


\linespread{1.1} % Line spacing

% Set up the header and footer
\pagestyle{fancy}
\lhead{\textbf{\hmwkClass -- \hmwkSubject \\ \hmwkTitle \\ \hmwkDocName}} % Top left header
\rhead{\includegraphics[width=10em]{../../images/logo_univ.png}}
\lfoot{\lastxmark} % Bottom left footer
\cfoot{} % Bottom center footer
\rfoot{Page\ \thepage\ / \pageref{LastPage}} % Bottom right footer
\renewcommand\headrulewidth{0.4pt} % Size of the header rule
\renewcommand\footrulewidth{0.4pt} % Size of the footer rule

\setlength{\headheight}{40pt}

\newcommand{\hmwkTitle}{Transchiffrement} % Assignment title
\newcommand{\hmwkClass}{Master 2 SSI } % Course/class
\newcommand{\hmwkAuthorName}{Jean-Baptiste Souchal} % Your name
\newcommand{\hmwkSubject}{Conduite de projet} % Subject
\newcommand{\hmwkDocName}{Compte-rendu Audit 20/12/13} % Document name

\newcommand{\version}{1.0} % Document version
\newcommand{\docDate}{20 Décembre 2013} % Document date
\newcommand{\checked}{Julien Bourdon} % Checker name

\makeatletter
\newcommand{\resettranslate}{\let\translate\@firstofone}
\makeatother

\definecolor{gris}{rgb}{0.95, 0.95, 0.95}

\title{
\vspace{2in}
\textmd{\textbf{\hmwkClass :\ \hmwkTitle}}\\
\normalsize\vspace{0.1in}\small{Due\ on\ \hmwkDueDate}\\
\vspace{0.1in}\large{\textit{\hmwkClassInstructor\ \hmwkClassTime}}
\vspace{3in}
}

\author{\hmwkAuthorName}
\date{} % Insert date here if you want it to appear below your name


\usepackage{amsmath}
\begin{document}
\newcount\startdate
\newcount\daynum
%\pgfcalendardatetojulian{2013-01-021}{\startdate}
\pagestyle{fancy}

\vspace*{5cm}
\begin{center}\textbf{\Huge{\hmwkDocName}}\end{center}
\vspace*{4.5cm}
	

\fcolorbox{black}{gris}{
\begin{minipage}{15cm}
\begin{tabularx}{10cm}{lXl}
	\bfseries{Version} & & \version\\
	& & \\
	\bfseries{Date} & & \docDate\\
	& & \\
	\bfseries{Rédigé par} & & \hmwkAuthorName \\
	& & \\
	\bfseries{Relu par} & & \checked \\
	& & \\
\end{tabularx}
\end{minipage}
}

\newpage

%Tableau de mises à jour


%La table des matières
%\clearpage
%\tableofcontents
%\clearpage

Liste des remarques à prendre en compte:
\begin{itemize}
  \item lister les tâches à suivre suite à chaque réunion
  \item avancer sur les documents pour le prochain audit de projet
  \item mettre dans la STB une liste des exigences du client
  \item mettre les études sur la partie MD5 dans la STB
  \item mettre un code sur chaque livrable pour le reporter sur les autres 
  documents (mettre en place une convention)
  \item faire valider les études par le client
  \item détailler chaque donnée utilisée (surtout pour la partie recherche MD5)
  \item exigence de qualité
\end{itemize}
~~\\
CDR:
~~\\
\begin{itemize}
  \item modification des responsabilités
  \item détailler le testeur, qui vérifie les tests ...
  \item définir l'environnement de test (Linux, Windows ...), navigateur ...
\end{itemize}
~~\\
DAL:
~~\\
\begin{itemize}
  \item détailler les cas d'utilisation
  \item finir les diagrammes de séquence
  \item utiliser des classes dans les diagrammes (puisque c'est un projet Java)
\end{itemize}
~~\\
PDD:
~~\\
\begin{itemize}
  \item faire un schéma sur l'interaction entre les acteurs principaux du projet
  \item développer l'introduction
  \item expliquer les besoins du cassage SSL, dans quel contexte (attaque par https...)
\end{itemize}
~~\\
STB:
~~\\
\begin{itemize}
  \item projet en 2 parties, faire la séparation dans la STB
  \item mettre à jour les terminologies et les règles utilisées
\end{itemize}
\end{document}