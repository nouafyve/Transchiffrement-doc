\documentclass[a4paper,11pt,french]{article}
\usepackage[utf8]{inputenc}

\usepackage[T1]{fontenc}
\usepackage[francais]{babel} 
\usepackage[top=2cm, bottom=2cm, left=2cm, right=2cm, includeheadfoot]{geometry} %pour les marges
\usepackage{lmodern}
\usepackage{pict2e}
\usepackage{fancyhdr} % Required for custom headers
\usepackage{lastpage} % Required to determine the last page for the footer
\usepackage{extramarks} % Required for headers and footers
\usepackage{graphicx} % Required to insert images
\usepackage{tabularx, longtable}
\usepackage{color, colortbl}
\usepackage{lscape}
%\usepackage[hidelinks]{hyperref}
\usepackage{longtable}
\usepackage{multirow}
\usepackage{rotating}
%\usepackage{pgfgantt}
%\usepackage{pgfcalendar}
%\usepackage{ifthen}
\usepackage{gensymb}


\linespread{1.1} % Line spacing

% Set up the header and footer
\pagestyle{fancy}
\lhead{\textbf{\hmwkClass -- \hmwkSubject \\ \hmwkTitle \\ \hmwkDocName}} % Top left header
\rhead{\includegraphics[width=10em]{../../images/logo_univ.png}}
\lfoot{\lastxmark} % Bottom left footer
\cfoot{} % Bottom center footer
\rfoot{Page\ \thepage\ / \pageref{LastPage}} % Bottom right footer
\renewcommand\headrulewidth{0.4pt} % Size of the header rule
\renewcommand\footrulewidth{0.4pt} % Size of the footer rule

\setlength{\headheight}{40pt}

\newcommand{\hmwkTitle}{Transchiffrement} % Assignment title
\newcommand{\hmwkClass}{Master 2 SSI } % Course/class
\newcommand{\hmwkAuthorName}{Jean-Baptiste Souchal} % Your name
\newcommand{\hmwkSubject}{Conduite de projet} % Subject
\newcommand{\hmwkDocName}{Compte rendu réunion 23/01/14} % Document name

\makeatletter
\newcommand{\resettranslate}{\let\translate\@firstofone}
\makeatother

\definecolor{gris}{rgb}{0.95, 0.95, 0.95}

\title{
\vspace{2in}
\textmd{\textbf{\hmwkClass :\ \hmwkTitle}}\\
\normalsize\vspace{0.1in}\small{Due\ on\ \hmwkDueDate}\\
\vspace{0.1in}\large{\textit{\hmwkClassInstructor\ \hmwkClassTime}}
\vspace{3in}
}

\author{\hmwkAuthorName}
\date{} % Insert date here if you want it to appear below your name

\usepackage{amsmath}
\begin{document}
\newcount\startdate
\newcount\daynum
%\pgfcalendardatetojulian{2013-01-021}{\startdate}
\pagestyle{fancy}

\vspace*{5cm}
\begin{center}\textbf{\Huge{\hmwkDocName}}\end{center}
\vspace*{4.5cm}

\newpage

%Tableau de mises à jour


%La table des matières
%\clearpage
%\tableofcontents
%\clearpage

\section{Personnes présentes :}
\begin{itemize}
  \item Magali Bardet
  \item Emile Générat
  \item Julien Bourdon
  \item Yves Nouafo
  \item Ouissem Hamdani
  \item Jean-Baptiste Souchal
\end{itemize}

\section{Sujets abordés :}
\begin{itemize}
  \item Etude sur l'acceptation d'une autorité par une personne.
  \item Les différentes méthodes envisagées pour faire accepter une autorité à 
  une personne (création d'un rapport d'analyse).
  \item S'inspirer du site igc.services.cnrs.fr pour faire accepter une 
  autorité.
  \item Justifier les choix techniques dans tous les documents de projet.
  \item STB : ajouter le détail du temps de réponse de la connexion causé par le proxy.
  \item PDD : lotissement - ajouter la mise en oeuvre de la forge de certificats MD5.
  \item Prévoir une présentation de l'ancement technique du projet toutes les semaines lors de la réunion avec le client.
  \item Documenter la génération de l'autorité ainsi que du certificat généré à la volée lors du transchiffrement.
  \item Réaliser une copie du flux arrivant sur le proxy, utilisation d'un fichier de log (avec effacement des logs selon une politique à définir), éventuellement transmis à un IDS (facultatif).
  \item Commencer à faire une recherche de collision MD5 simple dans un premier temps.
\end{itemize}

\section{Actions à prévoir pour la réunion de la semaine prochaine :}
\begin{itemize}
  \item Avancement sur la recherche de collision MD5.
  \item Collision MD5 simple.
  \item Présentation de l'avancement technique de l'application proxy.
\end{itemize}
~\\
La prochaine réunion est prévue le jeudi 30 à 15h30.

\end{document}