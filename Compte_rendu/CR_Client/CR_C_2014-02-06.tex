\documentclass[a4paper,11pt,french]{article}
\usepackage[utf8]{inputenc}

\usepackage[T1]{fontenc}
\usepackage[francais]{babel} 
\usepackage[top=2cm, bottom=2cm, left=2cm, right=2cm, includeheadfoot]{geometry} %pour les marges
\usepackage{lmodern}
\usepackage{pict2e}
\usepackage{fancyhdr} % Required for custom headers
\usepackage{lastpage} % Required to determine the last page for the footer
\usepackage{extramarks} % Required for headers and footers
\usepackage{graphicx} % Required to insert images
\usepackage{tabularx, longtable}
\usepackage{color, colortbl}
\usepackage{lscape}
%\usepackage[hidelinks]{hyperref}
\usepackage{longtable}
\usepackage{multirow}
\usepackage{rotating}
%\usepackage{pgfgantt}
%\usepackage{pgfcalendar}
%\usepackage{ifthen}
\usepackage{gensymb}


\linespread{1.1} % Line spacing

% Set up the header and footer
\pagestyle{fancy}
\lhead{\textbf{\hmwkClass -- \hmwkSubject \\ \hmwkTitle \\ \hmwkDocName}} % Top left header
\rhead{\includegraphics[width=10em]{../../images/logo_univ.png}}
\lfoot{\lastxmark} % Bottom left footer
\cfoot{} % Bottom center footer
\rfoot{Page\ \thepage\ / \pageref{LastPage}} % Bottom right footer
\renewcommand\headrulewidth{0.4pt} % Size of the header rule
\renewcommand\footrulewidth{0.4pt} % Size of the footer rule

\setlength{\headheight}{40pt}

\newcommand{\hmwkTitle}{Transchiffrement} % Assignment title
\newcommand{\hmwkClass}{Master 2 SSI } % Course/class
\newcommand{\hmwkAuthorName}{Émile GÉNÉRAT} % Your name
\newcommand{\hmwkSubject}{Conduite de projet} % Subject
\newcommand{\hmwkDocName}{Compte rendu réunion 30/01/14} % Document name

\makeatletter
\newcommand{\resettranslate}{\let\translate\@firstofone}
\makeatother

\definecolor{gris}{rgb}{0.95, 0.95, 0.95}

\title{
\vspace{2in}
\textmd{\textbf{\hmwkClass :\ \hmwkTitle}}\\
\normalsize\vspace{0.1in}\small{Due\ on\ \hmwkDueDate}\\
\vspace{0.1in}\large{\textit{\hmwkClassInstructor\ \hmwkClassTime}}
\vspace{3in}
}

\author{\hmwkAuthorName}
\date{} % Insert date here if you want it to appear below your name

\usepackage{amsmath}
\begin{document}
\newcount\startdate
\newcount\daynum
%\pgfcalendardatetojulian{2013-01-021}{\startdate}
\pagestyle{fancy}

\vspace*{5cm}
\begin{center}\textbf{\Huge{\hmwkDocName}}\end{center}
\vspace*{4.5cm}

\newpage

%Tableau de mises à jour


%La table des matières
%\clearpage
%\tableofcontents
%\clearpage

\section{Personnes présentes :}
\begin{itemize}
  \item Magali Bardet
  \item Emile Générat
  \item Julien Bourdon
  \item Yves Nouafo
  \item Ouissem Hamdani
  \item Jean-Baptiste Souchal
\end{itemize}

\section{Sujets abordés :}
Nous avons des problèmes pour établir une connexion TLS entre le client et le proxy.
Pour Madame Bardet, les points à vérifier/suggestions sont :
\begin{itemize}
\item s'assurer que le keystore contient les bonnes clés/certificats;
  \item le fichier cert.pem, qui doit contenir la concaténation des autorités de la chaîne;
  \item utiliser deux keystores;
  \item utiliser l'autorité intermédiaire faite en TP (pour ne pas avoir la clé privée de l'autorité racine) ;
  \item le certificat généré par le proxy pour un site doit avoir le même CN que le certificat du site distant.
  \item utiliser s\_client (si possible de le configurer avec un proxy)  
\end{itemize}

Pour l'acceptation des certificats d'autorité au sein des navigateurs, il faut chercher une solution sans devoir cliquer (problématique d'une flotte importante).

\section{Actions à prévoir pour la réunion de la semaine prochaine :}
\begin{itemize}
  \item Présenter notre projet à Madame Bardet mardi matin, pour vérifier si la partie cryptographique est correcte.
\end{itemize}
~\\
La prochaine réunion est prévue le jeudi 13 février dans l'après-midi.

\end{document}