À chaque demande de connexion SSL/TLS du client à un serveur, un faux certificat est envoyé au client. Il est généré si nécessaire.

Ce cas d'utilisation peut être décliné avec deux autorités, SSI-Sign, et SSI-Sign-Collision si nous avons trouvé une collision entre deux certificats.

Le proxy ne peut utiliser qu'une des deux autorités, sélectionnée manuellement lors de l'installation du proxy.

\begin{tabular}{|>{\columncolor[gray]{.8}}m{4cm}|m{12cm}|}
   \hline
   Description & Génération de Fake-Cert à destination du client. \\
   \hline
   Pré-conditions & \begin{itemize}
\item Le proxy est configuré sur le client.
\item Le proxy est lancé.
\item SSI-Cert est accepté par le client.
   \end{itemize}
 \\
   \hline
   Évènement déclenchant & Le client tente de se connecter à un serveur \\
   \hline
   Condition d'arrêt &  Fake-Cert et la chaîne de certification avec SSI-Cert sont envoyés au client. \\
   \hline
   Cas d'exception  &
\\
   \hline   
\end{tabular}

~\\

Description du flot d'évènements principal :

\begin{tabular}{|m{8cm}|m{8cm}|}
   \hline
  \rowcolor[gray]{.8} Acteur(s) & Système \\
   \hline
   1. Le client tente de se connecter à un serveur & \\
   \hline
    & 
2. 
Deux cas sont possibles :
\begin{itemize}
\item Si le certificat est déjà dans la base de données, on le récupère.
\item Sinon Fake-Cert est généré :
\begin{itemize}
\item Le issuer est SSI-Sign.
\item La clé publique est celle du bi-clé Fake-Cert, elle est constante pour tous les Fake-Cert. Nous connaissons la clé privée.
\end{itemize}
\end{itemize}
3. On envoie Fake-Cert au client. \\
   \hline
   4. Le client reconnaît Fake-Cert comme valide, car il est signé par SSI-Sign. &  \\
   \hline
\end{tabular}