\begin{tabular}{|>{\columncolor[gray]{.8}}m{4cm}|m{12cm}|}
   \hline
   Description & Utilisation d'un certificat généré à partir d'une collision. \\
   \hline
   Pré-conditions & \begin{itemize}
\item  L'étude a permis d'obtenir un deuxième certificat de même haché MD5, mais dont on connaît la clé privée associée.
  \item Le client possède l'autorité attaquée.
   \end{itemize} \\
   \hline
   Évènement déclenchant & Un client demande une connexion. \\
   \hline
   Condition d'arrêt & Le certificat Fake-Cert est reconnu valide par le client. \\
   \hline
   Cas d'exception  &  \\
   \hline   
\end{tabular}

~\\

Description du flot d'évènements principal :

\begin{tabular}{|m{8cm}|m{8cm}|}
   \hline
  \rowcolor[gray]{.8} Acteur(s) & Système \\
   \hline
   1. Le client demande une connexion. & \\
   \hline
&    2. Le proxy génère Fake-Cert et le signe avec SSI-Sign-Collision.  \\
   \hline
 Le client reconnaît Fake-Cert comme valide, car la signature de SSI-Cert-Collision est valide (duppliquée d'un certificat de même haché) & \\
\hline
\end{tabular}