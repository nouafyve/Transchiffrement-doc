
Une fois le certificat généré, le client établit une connexion TLS/SSL et envoie sa requête au proxy de manière chiffrée, le proxy déchiffre, la garde dans les logs (éventuellement) puis rechiffre cette requête pour le serveur (avec une autre clé). Le but de ce processus est d'être transparent donc très rapide.

Ici, le cas décrit le transchiffrement d'un paquet du client vers le serveur. Le processus est inversé lors de la réponse du serveur vers le client .

\begin{tabular}{|>{\columncolor[gray]{.8}}m{4cm}|m{12cm}|}
   \hline
   Description & Déchiffrement / Rechiffrement de la requête du client pour l'envoyer au serveur et de la réponse du serveur pour l'envoyer au client. \\
   \hline
   Pré-conditions & Certificat généré et deux connexions TLS/SSL ouvertes\\
   \hline
   Évènement déclenchant & Réception d'un paquet Application-Data par le proxy. \\
   \hline
   Condition d'arrêt & Paquet rédigé et transfert d’envoi terminé \\
   \hline
   Cas d'exception  & Renégociation TLS/SSL \\
   \hline   
\end{tabular}

~\\

Description du flot d'évènements principal :

\begin{tabular}{|m{8cm}|m{8cm}|}
   \hline
  \rowcolor[gray]{.8} Acteur(s) & Système \\
   \hline
   1. Envoi d'une requête chiffrée par le client & \\
   \hline
    &
2. Déchiffrement de la requête

3. Log de cette requête

4. Rechiffrement de la requête pour le serveur

5. Envoi de cette requête au serveur


8. Réception de la réponse chiffrée

9. Déchiffrement de la réponse

10. Log de la réponse\\
\hline
\end{tabular}