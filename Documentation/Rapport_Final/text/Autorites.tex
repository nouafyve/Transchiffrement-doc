De nos jours, nous avons besoin d'authentifier les sites auxquels on veut accéder pour être sur que les données cruciales que nous manipulons ne tombent pas en de mauvaises mains. Pour ce faire, on utilise des certificats signés par des autorités en qui on peut avoir confiance.
Ces autorités sont répertoriées dans nos machines et plus précisément dans nos navigateurs par le biais de certificats d'autorité.
Nous allons maintenant voir comment on peut faire accepter l'installation d'une nouvelle autorité sur sa machine en quelques clics.




\subsection{Installation par un administrateur mal intentionné}
Nous prenons, ici, le cas d'un administrateur ayant un accès à tous les ordinateurs d'une entreprise.
Cette personne veut faire accepter une autorité de certification dont il est le propriétaire pour pouvoir lire tous les paquets qui transitent et surtout les chiffrés.
Nous allons expliquer comment faire pour les principaux navigateurs utilisés sous linux.
Voilà un tableau comparatif des pourcentages d'utilisation des navigateurs : 

Moyenne générale (Janvier 2014) :
\begin{itemize}
\item{Chrome} 		43.67\%
\item{Internet Explorer}		 22,85\%
\item{Firefox} 		18,9\%
\item{Safari} 		9.73\%
\item{Opera}	 		1,3\%	
\item{Autres} 		3.55\%
\end{itemize}	


D'après le site : \begin{verbatim}
http://gs.statcounter.com/
\end{verbatim}

Nous allons privilégier Firefox et Chrome, cela nous permettra de viser plus de 60\% des machines.

Nous pouvons imaginer que l'administrateur installe ces navigateurs sur les machines de l'entreprise.
\newpage



\subsection{Installation par l'utilisateur}
Dans ce cas, peu importe le navigateur du client, on lui demande d'installer un certificat sur sa machine.

Lorsqu'un utilisateur


\subsection{Collision MD5}
Dans ce cas, aucune opération manuelle n'est nécessaire, ni par l'utilisateur, ni par l'administrateur.


