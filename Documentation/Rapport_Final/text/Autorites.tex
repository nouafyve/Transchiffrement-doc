De nos jours, nous avons besoin d'authentifier les sites auxquels on veut accéder pour être sur que les données cruciales que nous manipulons ne tombent pas en de mauvaises mains. Pour ce faire, on utilise des certificats signés par des autorités en qui on peut avoir confiance.
Ces autorités sont répertoriées dans nos machines et plus précisément dans nos navigateurs par le biais de certificats d'autorité.
Nous allons maintenant voir comment on peut faire accepter l'installation d'une nouvelle autorité sur sa machine en quelques clics.




\subsection{Installation par un administrateur mal intentionné}
Nous prenons, ici, le cas d'un administrateur ayant un accès à tous les ordinateurs d'une entreprise.
Cette personne veut faire accepter une autorité de certification dont il est le propriétaire pour pouvoir lire tous les paquets qui transitent et surtout les chiffrés.
Nous allons expliquer comment faire pour les principaux navigateurs utilisés sous linux.

Nous pouvons imaginer que l'administrateur installe ces navigateurs sur les machines de l'entreprise.
\newpage



\subsection{Installation par l'utilisateur}
Dans ce cas, peu importe le navigateur du client, on lui demande d'installer un certificat sur sa machine.

Lorsqu'un utilisateur


\subsection{Collision MD5}
Dans les trois précédentes parties nous avons vu comment imposer ou installer une 
autorité de certification dans le navigateur web d'un utilisateur. Cette 
autorité va permettre l'acceptation du certificat proxy.

La collision de certificat de type MD5 permet de générer un certificat ayant 
exactement la même signature que le certificat original du serveur web visé, et 
donc totalement valide auprès du navigateur web. A chaque échange de certificat, 
la signature de ce dernier est vérifiée auprès de son autorité de certification, 
la faux certificat ayant une signature identique sera reconnu comme valide.
~~\\

Le but de la recherche de collision sur les certificats de type MD5 est donc d'éviter les
procédures mis en place et décrites dans ces trois dernières parties. La forge d'un faux 
certificat à partir d'un certificat serveur, reconnu par une autorité de 
certification valide aux yeux de tous les navigateurs, va permettre à notre 
proxy d'être reconnu comme un vrai serveur web. De ce fait il ne sera plus 
nécessaire de faire accepter notre autorité au navigateur du client.

Dans ce cas, aucune opération manuelle n'est nécessaire, ni par l'utilisateur, ni par l'administrateur.
Cela simplifie grandement la mise en place du proxy et limite également la
détection du proxy par l'utilisateur.

