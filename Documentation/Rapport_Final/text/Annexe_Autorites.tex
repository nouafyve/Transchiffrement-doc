\subsection{Mozilla Firefox}

Tout d'abord, l'administrateur ouvre Firefox puis clique sur Edit > Preferences.

\includegraphics[width=\textwidth]{images_autorites/OngletPref.png}
\newpage
Ensuite, il choisit Advanced > Certificates > View Certificates.

\includegraphics[width=\textwidth]{images_autorites/OngletCert.png}
\newpage
Il va ensuite dans Authorities > Import.

\includegraphics[width=\textwidth]{images_autorites/OngletCA.png}
\newpage
Il choisit ensuite le certificat de l'autorité qu'il veut installer puis valide.

\includegraphics[width=\textwidth]{images_autorites/OngletImport.png}
\newpage
Une fenêtre s'ouvre et propose de faire confiance à cette autorité pour 3 types de Certificats. L'administrateur coche les 3 cases pour que son autorité soit reconnue valide sur tous les types puis clique sur ok.

\includegraphics[width=\textwidth]{images_autorites/OngletConfirm.png} 


Voilà, l'autorité est installée et tous les certificats signés par cette autorité seront reconnus comme valides.
\newpage
\subsection{Chrome}

La démarche est très similaire à celle de firefox.

Tout d'abord, l'administrateur va dans Modifier > Préférences

\includegraphics[width=\textwidth]{images_autorites/ChromePref.png} 
\newpage

Puis il clique sur Afficher les paramètres avancés.

\includegraphics[width=\textwidth]{images_autorites/ChromeAvance.png} 
\newpage

Ensuite, dans la partie HTTPS/SSL, il clique sur Gérer les certificats

\includegraphics[width=\textwidth]{images_autorites/ChromeCert.png} 
\newpage

Il se déplace dans Autorités et clique sur Importer

\includegraphics[width=\textwidth]{images_autorites/ChromeCA.png} 
\newpage

Il choisit le certificat de l'autorité et valide.

\includegraphics[width=\textwidth]{images_autorites/ChromeImport.png} 
\newpage

Enfin, il coche les 3 cases et clique sur ok pour finaliser l'installation de l'autorité.

\includegraphics[width=\textwidth]{images_autorites/ChromeValide.png} 
\newpage

\section{Forcer l'acceptation de l'autorité par un client}
Dans cette section, nous forçons l'utilisateur a accepter notre autorité de certification. Si il ne l'a pas, il ne pourra pas naviguer sur internet.
Nous proposons donc un lien pour récupérer le certificat d'autorité sur lequel il faut cliquer.

\includegraphics[width=\textwidth]{images_autorites/Page.png} 
\newpage

Ensuite, la fenêtre de validation s'ouvre et le client doit cocher les cases puis valider.

\includegraphics[width=\textwidth]{images_autorites/Cert.png} 
\newpage

A ce stade, l'autorité est installée et si le client veut retenter de l'installer, un message le prévient qu'il a déjà fini l'installation.

\includegraphics[width=\textwidth]{images_autorites/Alerte.png}

On peut voir qu'en seulement quelques clics, l'utilisateur installe une autorité dont il ne connaît rien et qui peut être utilisée pour déchiffrer toutes ses informations personnelles. 