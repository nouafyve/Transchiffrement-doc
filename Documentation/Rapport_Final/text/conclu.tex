Tout d'abord ce projet nous a permis de mettre en évidence, dans le cadre d'un 
réseau interne, un risque de confidentialité majeur pour les utilisateurs. Comme
nous l'avons démontré, la mise en place d'un proxy de transchiffrement SSL sur un réseau permet 
une lisibilité totale des échanges utilisant le protocole SSL.

D'autre part, et de façon contradictoire, un tel proxy permet de lutter plus 
efficacement contres les logiciels malveillants qui utilisent le protocole SSL 
pour pénétrer au sein d'un réseau.
~~\\

Il est fort probable qu'un certains nombres d'entreprises utilisent ce type de proxy 
pour surveiller le trafic entrant et sortant de leurs réseaux, mais également 
dans le but de surveiller leurs utilisateurs sans avoir leurs approbations. 

Dans le cas ou une entreprise utilise un proxy de transchiffrement SSL pour 
surveiller le trafic, elle ne pourra faire autrement que de filtrer n'importe 
quel échange sécurisé et à fortiori récupérer des informations personnelles sur 
les utilisateurs de son réseau. Dans ce cas de figure, les utilisateurs 
devraient en être informé et l'entreprise devrait être dans l'obligation de ne 
faire aucun usage de ces informations. Dans les faits, aucune étude ne permet de 
prouver de tel conventions entre une entreprise et ces utilisateurs.
~~\\

Ce projet nous a  également appris que la confiance que nous mettons en certaines autorités doit être mûrement réfléchie.
En effet, si une personne de confiance comme un administrateur système, ou une personne malveillante arrive à
configurer un proxy de transchiffrement sur un réseau comme nous l'avons fait et que les utilisateurs
s'empressent d'accepter une nouvelle autorité inconnue, alors toute la sécurité est remise en cause.

Notre expérience personnelle tend à prouver qu'il est facile de faire accepter 
une autorité à un utilisateur lambda sans aucune connaissance particulière en 
informatique et encore moins en sécurité informatique.

Grâce à la négligence des utilisateurs et du proxy que nous avons développé, nous sommes en mesure de récupérer toutes les données échangées
via le protocole HTTPS, entre autre récupérer les accès d'un utilisateur à ses comptes en ligne, de consulter les mails, qui contiennent souvent
des informations personnelles.
~~\\

Dans un autre registre, la gestion de projet, qui a occuper les premières semaines de notre 
projet, nous a permis de construire une vision d'ensemble du projet à l'aide de document de 
spécifications.
La gestion de projet, qui pour nous est encore récent dans notre conception de projet, a permis de faire évoluer
nos méthodes de réflexion et d'analyse d'un sujet de manière professionnelle et adaptable en fonction des besoins du client.
