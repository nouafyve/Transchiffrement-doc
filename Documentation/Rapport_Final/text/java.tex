\documentclass[a4paper,11pt,french]{article}
\usepackage[utf8]{inputenc}

\usepackage[T1]{fontenc}
\usepackage[francais]{babel} 
\usepackage[top=2cm, bottom=2cm, left=2cm, right=2cm, includeheadfoot]{geometry} %pour les marges
\usepackage{lmodern}
\usepackage{pict2e}
\usepackage{fancyhdr} % Required for custom headers
\usepackage{lastpage} % Required to determine the last page for the footer
\usepackage{extramarks} % Required for headers and footers
\usepackage{graphicx} % Required to insert images
\usepackage{tabularx, longtable}
\usepackage{color, colortbl}
\usepackage{lscape}
%\usepackage[hidelinks]{hyperref}
\usepackage{longtable}
\usepackage{multirow}
\usepackage{rotating}
\usepackage{gensymb}
\usepackage{soulutf8}

\linespread{1.1} % Line spacing

% Set up the header and footer
\pagestyle{fancy}
\lhead{\textbf{\hmwkClass -- \hmwkSubject \\ \hmwkTitle \\ \hmwkDocName}} % Top left header
\rhead{\includegraphics[width=10em]{../../images/logo_univ.png}}
\lfoot{\lastxmark} % Bottom left footer
\cfoot{} % Bottom center footer
\rfoot{Page\ \thepage\ / \pageref{LastPage}} % Bottom right footer
\renewcommand\headrulewidth{0.4pt} % Size of the header rule
\renewcommand\footrulewidth{0.4pt} % Size of the footer rule

\setlength{\headheight}{40pt}

\newcommand{\hmwkTitle}{Transchiffrement} % Assignment title
\newcommand{\hmwkClass}{Master 2 SSI } % Course/class
\newcommand{\hmwkAuthorName}{Émile GÉNÉRAT} % Your name
\newcommand{\hmwkSubject}{Conduite de projet} % Subject
\newcommand{\hmwkDocName}{Spécification Technique du Besoin} % Document name

\newcommand{\version}{1.0} % Document version
\newcommand{\docDate}{} % Document date
\newcommand{\checked}{Jean-Baptiste SOUCHAL} % Checker name
\newcommand{\approved}{} % Approver name

\makeatletter
\newcommand{\resettranslate}{\let\translate\@firstofone}
\makeatother

\definecolor{gris}{rgb}{0.95, 0.95, 0.95}

\title{
\vspace{2in}
\textmd{\textbf{\hmwkClass :\ \hmwkTitle}}\\
\normalsize\vspace{0.1in}\small{Due\ on\ \hmwkDueDate}\\
\vspace{0.1in}\large{\textit{\hmwkClassInstructor\ \hmwkClassTime}}
\vspace{3in}
}

\author{\hmwkAuthorName}
\date{} % Insert date here if you want it to appear below your name


\usepackage{amsmath}
\begin{document}
\newcount\startdate
\newcount\daynum
%\pgfcalendardatetojulian{2013-01-021}{\startdate}
\pagestyle{fancy}

\vspace*{5cm}
\begin{center}\textbf{\Huge{\hmwkDocName}}\end{center}
\vspace*{4.5cm}
	

\fcolorbox{black}{gris}{
\begin{minipage}{15cm}
\begin{tabularx}{10cm}{lXl}
	\bfseries{Version} & & \version\\
	& & \\
	\bfseries{Date} & & \docDate\\
	& & \\
	\bfseries{Rédigé par} & & \hmwkAuthorName \\
	& & \\
	\bfseries{Relu par} & & \checked \\
	& & \\
	\bfseries{Approuvé par} & & \approved \\
	& & \\
\end{tabularx}
\end{minipage}
}

\newpage

%Tableau de mises à jour
\vspace*{1cm}
\begin{center}
\textbf{\huge{Versions}}\\
\vspace*{3cm}
	\begin{tabularx}{16cm}{|c|c|X|}
	\hline
	\bfseries{Version} & \bfseries{Date} & \bfseries{Modifications réalisées}\\
	\hline
	1.0 &  & Création\\
	\hline
	\end{tabularx}
\end{center}

%La table des matières
\clearpage
\tableofcontents
\clearpage


javax.net.ssl.keyStore- Location of the Java keystore file containing an application process's own certificate and private key. On Windows, the specified pathname must use forward slashes, /, in place of backslashes.

javax.net.ssl.keyStorePassword - Password to access the private key from the keystore file specified by javax.net.ssl.keyStore. This password is used twice: To unlock the keystore file (store password), and To decrypt the private key stored in the keystore (key password).

javax.net.ssl.trustStore - Location of the Java keystore file containing the collection of CA certificates trusted by this application process (trust store). On Windows, the specified pathname must use forward slashes, /, in place of backslashes.

If a trust store location is not specified using this property, the SunJSSE implementation searches for and uses a keystore file in the following locations (in order):

\begin{verbatim}
$JAVA_HOME/lib/security/jssecacerts
$JAVA_HOME/lib/security/cacerts
\end{verbatim}
javax.net.ssl.trustStorePassword - Password to unlock the keystore file (store password) specified by javax.net.ssl.trustStore.

javax.net.ssl.trustStoreType - (Optional) For Java keystore file format, this property has the value jks (or JKS). You do not normally specify this property, because its default value is already jks. javax.net.debug To switch on logging for the SSL/TLS layer, set this property to ssl.



A keystore contains private keys, and the certificates with their corresponding public keys.

A truststore contains certificates from other parties that you expect to communicate with, or from Certificate Authorities that you trust to identify other parties.


A keystore contains a private key. You only need this if you are a server, or if the server requires client authentication.

A truststore contains CA certifcates to trust. If your server’s certificate is signed by a recognized CA, the default truststore that ships with the JRE will already trust it (because it already trusts trustworthy CAs), so you don’t need to build your own, or to add anything to the one from the JRE.







In SSL handshake purpose of trustStore is to verify credentials and purpose of keyStore is to provide credential.

keyStore in Java stores private key and certificates corresponding to there public keys and require if you are SSL Server or SSL requires client authentication.

TrustStore stores certificates from third party, your Java application communicate or certificates signed by CA(certificate authorities like Verisign, Thawte, Geotrust or GoDaddy) which can be used to identify third party.

TrustManager determines whether remote connection should be trusted or not i.e. whether remote party is who it claims to and KeyManager decides which authentication credentials should be sent to the remote host for authentication during SSL handshake.

If you are an SSL Server you will use private key during key exchange algorithm and send certificates corresponding to your public keys to client, this certificate is acquired from keyStore. On SSL client side, if its written in Java, it will use certificates stored in trustStore to verify identity of Server. SSL certificates are most commonly comes as .cer file which is added into keyStore or trustStore by using any key management utility e.g. keytool.


\end{document}