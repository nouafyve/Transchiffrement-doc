\paragraph{Environnement de test}
Les tests ont été effectués dans un environnement de type Unix (Ubuntu, Mac OS 
X) pour la partie utilisateur.~~\\

Les machines utilisées pour les tests ont requis une connexion internet ainsi qu'un
navigateur web supportant l'usage de certificats de type MD5. Aucune restriction sur la configuration matérielle des machines.~~\\

L'analyse des échanges entre les différentes entités a été faite à l'aide de l'outil Wireshark.
\begin{itemize}
\item une machine virtuelle « proxy » où est installé le proxy 
\item une machine « client » qui joue le rôle d'utilisateur lambda sur le réseau
\end{itemize}

\paragraph{Stratégie de tests}
Un test a été validé lorsqu’il répondait à l’exigence fonctionnelle à laquelle il était lié.~~\\

Un test non validé a fait l’objet d’un retour vers le(s) développeur(s) du module concerné.
Chaque test non validé a impliqué la correction, par le(s) développeur(s), du module concerné dans
un délai raisonnable en fonction du planning et du plan de développement mis en place par le chef de
projet.~~\\

Après correction, le module a de nouveau été testé. Après avoir effectué tous les tests, les
résultats ont été envoyés au chef de projet.

\paragraph{Gestion des anomalies}
Le responsable du module concerné par l’anomalie a été chargé de la résoudre dans un délai raisonnable en fonction de la gravité de cette anomalie pour le fonctionnement global
  de l’application. ~~\\
    
  Le délai de correction a dû être inclus dans le planning du développeur en fonction
  du plan de développement émis par le chef de projet.
  
\paragraph{Explications}
Nous avons tout d'abord fait nos tests en local pour plus de simplicité, c'est pourquoi nous avions un serveur web gérant une partie en HTTP et une autre en HTTPS.~~\\

Une fois le proxy fonctionnant sur notre serveur web en local, nous avons testé différents sites amenant chacun leurs problèmes.~~\\

Les tests ont donc été centrés sur certains sites HTTPS (le HTTP n'étant pas un problème).
Par exemple, nous avons pu avoir accès au numéro de compte ainsi qu'au mot de passe d'un membre de notre groupe.