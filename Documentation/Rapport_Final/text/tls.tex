\subsubsection{définition}
le protocole TLS/SSL est un protocole de sécurisation des échanges sur internet.Il fonctionne suivant un mode client/serveur
TLS/SSL assure l'authentification, la confidentialité et l'intégrité.
\subsubsection{Fragmentation}
la fragmentation des blocs d'informations en des "record" TLS/SSL porte sur des données 2\^\ 14 octets ou moins.
\subsubsection{Fonction HMAC}
Pour protéger l'intégrité du message TLS/SSL utilise le code MAC,
le chiffrement utilise une construction HMAC qui est basée sur une fonction de hashage.
\subsubsection{Le protocole "record" TLS/SSL}
\begin{itemize}
\item L'envoie:
TLS/SSL prend les messages à transmettre, fragmente les données en des blocs, compresse les données (optionnel), applique le MAC, crypte et transmis le résultat. 
\item
La récéption:A la réception, les données sont déchiffrées, vérifiées, décompréssées, rassemblées, et aprés ils sont livrées à des clients de niveau supérieur.
\end{itemize}
le protocle handshake utilise ce genre d'échange de message.
\subsubsection{Le protocole handshake:}

le protocole handshake est responsable pour la négociation d'une session.Cette session se décompose des éléments suivants:
\begin{itemize}
\item session identifier:ce sont des séquence de bit arbitraires choisis par le serveur pour identifier une session active.

\item peer certificate: ce champ contient le certificat X509v3 .l'état de cet élément peut être nulle.
\item compression method: l'algorithme utilisé pour compresser les données avant le chiffrement.
\item cipher spec:spécifie la fonction utilisées pour générer des clés, algorithme de chiffrement, algorithme MAC et les attributs cryptographiques.
\item master secret:le serect partagés entre le client et le serveur.
\item is resumable: un flag qui indique si la session peut etre utilisés pour initialiser une nouvelle connection ou non.\\
\end{itemize}
le protocole handshake utilise les étapes suivantes:
\begin{itemize}
\item échanger un message "hello messages" pour se mettre d'accord sur l'algorithme.
\item échanger des paramètres cryptographiques pour accepter le secret entre le client et le serveur.
\item échanger les certificats et des informations cryptographiques pour permettre au client et au serveur de s'authentifier. 
\item générer un secret du secret préliminaire et échange des valeurs aléatoires.
\item Fournir les paramètres de sécurité.
\item permettre au client et au serveur de vérifier qu'ils ont le m\^eme paramètre de sécurité et que le handshake a eu lieu sans altération d'un attaquant.
\end{itemize}
\paragraph{Hello message}
le message "Hello message" est utilisé pour renforcer les échanges sécurisés entre le client et le serveur.
\subsubsection {Hello request:}
"Hello request" est une notification pour que le client doit commencer la négaciation d'une connexion.Le message "hello request" peut \^etre envoyer à n'importe quel moment.

\subsubsection {Client hello}
Lors de la première connexion du client au serveur, il est nécessaire d'envoyer un "client hello" comme premier message.Le client peut aussi envoyer un "client hello" comme une réponse de "hello request".
\subsubsection {Server hello}
Le serveur va envoyer ce message suite a une réponse du message "client hello" si l'algorithme est accepté si non envoie handshake "failure alert".
\subsubsection {Signature algorithms}
le client l'utilise pour indiquer au serveur quel paire signature/hash peut \^etre utilisé.
\paragraph{Server Certificate}
le serveur doit envoyer un certificat.
ce certificat doit être appropriée pour  la suite de chiffrement et toutes les extensions négociées.
\paragraph{Server key exchange message}
envoyer par le serveur aprés le message "server certificate" et que si cette dernière ne contient pas assez de données pour permettre au client l'échange du secret.
\paragraph{Certificate Request}
Ce message, s'il est envoyé va suivre le message "ServerKeyExchange"
\paragraph{Server Hello Done}
envoyerpaar le serveur pour indiquer la fin du message "Server Hello Done". Le serveur attend une réponse du client.
\paragraph{Client Certificate}
ce message est envoyé seulement si le serveur demande un certificat.
\paragraph{Client Key Exchange Message}
avec ce message le secret est définie soit transmission directe avec RSA ou par Diffie-Hellman pour que les deux entités se mettent d'accord sur le même code secret.
\paragraph{Certificate Verify}
il est utlisé pour impliquer une vérification explicit du certificat client.
\paragraph{Finished}
envoyer aprés un échange chiffré pour vérifier que l'échange et l'authentification est passé entre les deux entités utilisant le protocole handshake.
