\subsection{Authentification client}

L'authentification client sur un site web avec l'utilisation de certificats permet de créer une authentification forte.
Ce type d'authentification est beaucoup plus sûr qu'une authentification par 
login et mot de passe, trop facilement trouvable par un attaquant.

Cependant, il est rare qu'un site web sécurisé avec HTTPS utilise de l'authentification 
client. Pour gérer une authentification client, les serveurs doivent être configurés d'une certaine manière. 
Pour exemple le site des impôts Français a essayer de mettre en place ce 
type d'authentification mais cela c'est révélé être un échec dû à la difficulté 
apparente pour une majeur partie des utilisateurs.

Dans le cadre de notre projet, l'utilisation d'une authentification client entre le proxy et le serveur web n'est 
pas réalisable du fait que nous ne gérons pas les serveurs des sites web, et il 
est impossible d'imposer à un site une authentification client alors qu'il 
n'implémente pas cette méthode.

D'autre part si le site web demandé par un client demande l'authentification de 
ce dernier, elle ne sera pas possible à implémentée. Un certificat client est 
généré et signé par l'AC du serveur, or le proxy établit une connexion SSL avec 
le client en utilisant ca propre AC, et ne pourra donc pas reconnaitre le 
certificat du client comme valide, lors de la création du contexte SSL avec le client. 

