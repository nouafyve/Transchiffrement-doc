\documentclass[a4paper,11pt,french]{report}
\usepackage[utf8]{inputenc}

\usepackage[T1]{fontenc}
\usepackage[francais]{babel} 
\usepackage[top=2cm, bottom=2cm, left=2cm, right=2cm, includeheadfoot]{geometry} %pour les marges
\usepackage{lmodern}
\usepackage{pict2e}
\usepackage{fancyhdr} % Required for custom headers
\usepackage{lastpage} % Required to determine the last page for the footer
\usepackage{extramarks} % Required for headers and footers
\usepackage{graphicx} % Required to insert images
\usepackage{tabularx, longtable}
\usepackage{color, colortbl}
\usepackage{lscape}
%\usepackage[hidelinks]{hyperref}
\usepackage{longtable}
\usepackage{multirow}
\usepackage{rotating}
\usepackage{gensymb}
\usepackage{tikz}
\usepackage{pgfplots}


\linespread{1.1} % Line spacing

% Set up the header and footer
\pagestyle{fancy}
\lhead{\textbf{\hmwkClass -- \hmwkSubject \\ \hmwkTitle \\ \hmwkDocName}} % Top left header
\rhead{\includegraphics[width=10em]{../../images/logo_univ.png}}
\lfoot{\lastxmark} % Bottom left footer
\cfoot{} % Bottom center footer
\rfoot{Page\ \thepage\ / \pageref{LastPage}} % Bottom right footer
\renewcommand\headrulewidth{0.4pt} % Size of the header rule
\renewcommand\footrulewidth{0.4pt} % Size of the footer rule

\setlength{\headheight}{40pt}

\newcommand{\hmwkTitle}{Projet transchiffrement SSL/TLS} % Assignment title
\newcommand{\hmwkClass}{Master 2 SSI } % Course/class
\newcommand{\hmwkAuthorName}{Julien BOURDON, Émile GÉNÉRAT, Jean-Baptiste SOUCHAL} % Your name
\newcommand{\hmwkSubject}{} % Subject
\newcommand{\hmwkDocName}{Rapport final} % Document name

\newcommand{\version}{1.0} % Document version
\newcommand{\docDate}{20 février 2014} % Document date
\newcommand{\checked}{} % Checker name
\newcommand{\approved}{} % Approver name

\makeatletter
\newcommand{\resettranslate}{\let\translate\@firstofone}
\makeatother

\definecolor{gris}{rgb}{0.95, 0.95, 0.95}

\title{
\vspace{2in}
\textmd{\textbf{\hmwkClass :\ \hmwkTitle}}\\
\normalsize\vspace{0.1in}\small{Due\ on\ \hmwkDueDate}\\
\vspace{0.1in}\large{\textit{\hmwkClassInstructor\ \hmwkClassTime}}
\vspace{3in}
}

\author{\hmwkAuthorName}
\date{} % Insert date here if you want it to appear below your name


\usepackage{amsmath}
\begin{document}
\newcount\startdate
\newcount\daynum
%\pgfcalendardatetojulian{2013-01-021}{\startdate}
\pagestyle{fancy}

\vspace*{5cm}
\begin{center}\textbf{\Huge{\hmwkDocName}}\end{center}
\vspace*{4.5cm}
	

\fcolorbox{black}{gris}{
\begin{minipage}{15cm}
\begin{tabularx}{10cm}{lXl}
	\bfseries{Version} & & \version\\
	& & \\
	\bfseries{Date} & & \docDate\\
	& & \\
	\bfseries{Rédigé par} & & \hmwkAuthorName \\
	& & \\
	\bfseries{Relu par} & & \checked \\
	& & \\
	\bfseries{Approuvé par} & & \approved \\
	& & \\
\end{tabularx}
\end{minipage}
}

\newpage

%La table des matières
\clearpage
\tableofcontents
\clearpage

\chapter{Présentation du projet}


\section{Introduction}
Pour garantir la confidentialité du trafic internet, les sites ont de plus en plus souvent recours au chiffrement des échanges. Ce chiffrement s'effectue de bout en bout, du client jusqu'au serveur.
Ainsi, un intrus qui intercepte les connexions ne peut pas lire les paquets qui transitent.


Fréquemment, les entreprises analysent le trafic entrant et sortant de leur réseau. Cette analyse du contenu des paquets permet par exemple de chercher la présence de virus.


Le but du projet est de fournir une solution de transchiffrement, qui permette d'analyse en clair au sein du proxy les paquets, qu'ils soient issus d'une connexion en clair ou chiffrée. Dans ce dernier cas, il faut établir une connexion chiffrée vers le client, et une autre vers le serveur distant. Ce système permet de faire une attaque de type "Man In The Middle".


Pour cela, nous aurons besoin d'une autorité de certification reconnue par le client pour générer des certificats.
Il existe plusieurs méthodes, dont l'étude des collisions MD5.
Le but de la collision revient à générer un deuxième certificat dont la clé publique est différente, mais qui a la même signature MD5.




\section{Spécifications}

\subsection{Cas d'utilisation}
\includegraphics[width=0.8\textwidth]{../../STB/images/cas_utilisation.pdf}

\subsection{Schéma du système}
\includegraphics[width=0.8\textwidth]{../../STB/images/schema_autorites.pdf}


\section{Gestion de projet}

\subsection{L'équipe}

Pour mener à bien ce projet, nous sommes une équipe de 5 étudiants.

Nous nous sommes répartis autour des deux parties principales parties du projet.
\begin{itemize}
  \item Un proxy de transchiffrement SSL.
  \item La rechercher de collision sur des certificats de type MD5.

\end{itemize}

\subsection{Présentation des livrables}

Les fonctionnalités finales attendues sont :
\begin{itemize}
\item Une application servant de proxy réalisant du transchiffrement SSL.
\item Un dossier de recherche sur les collisions de certificats de type MD5, ainsi qu'une mise en oeuvre 
par un algorithme de recherche.
\end{itemize}

\subsection{Évolution du planning et des priorités}

Nous avons rencontré des difficultés dans la réalisation du projet.
En accord avec le client, nous avons privilégié la réalisation du programme applicatif de proxy.

\subsection{Gestion des risques}

Nous avons établi une liste des risques qui pouvaient menacer la réussite du projet.

Ces risques ont fait l'objet d'une attention particulière, tout au long du projet.

\chapter{Proxy SSL/TLS}
\section{Analyse}

\subsection{Cibles visées par l'attaque}

\subsection{Protocole TLS}
\subsubsection{Définition}
Le protocole TLS/SSL est un protocole de sécurisation des échanges sur internet. Il fonctionne suivant un mode client/serveur
TLS/SSL assure l'authentification, la confidentialité et l'intégrité.
\subsubsection{Fragmentation}
La fragmentation des blocs d'informations en des "record" TLS/SSL porte sur des données 2\^\ 14 octets ou moins.
\subsubsection{Fonction HMAC}
Pour protéger l'intégrité du message TLS/SSL utilise le code MAC,
le chiffrement utilise une construction HMAC qui est basée sur une fonction de hachage.
\subsubsection{Le protocole "record" TLS/SSL}
\begin{itemize}
\item L'envoi:
TLS/SSL prend les messages à transmettre, fragmente les données en des blocs, compresse les données (optionnel), applique le MAC, chiffre et transmet le résultat. 
\item
La réception:
A la réception, les données sont déchiffrées, vérifiées, décompressées, rassemblées, puis elles sont livrées à des clients de niveau supérieur.
\end{itemize}
le protocole handshake utilise ce genre d'échange de message.
\subsubsection{Le protocole handshake:}

Le protocole handshake est responsable pour la négociation d'une session.
Cette session se compose des éléments suivants:
\begin{itemize}
\item Session identifier: séquences de bit arbitraires choisis par le serveur pour identifier une session active.
\item Peer certificate: ce champ contient le certificat X509v3. 
\item Compression method: l'algorithme utilisé pour compresser les données avant le chiffrement.
\item Cipher spec: spécifie la fonction utilisée pour générer des clés, l'algorithme de chiffrement, l'algorithme MAC et les attributs cryptographiques.
\item Master secret: le secret partagé entre le client et le serveur.
\item Is resumable: un flag qui indique si la session peut être utilisée pour initialiser une nouvelle connexion ou non.\\
\end{itemize}
Le protocole handshake utilise les étapes suivantes:
\begin{itemize}
\item Echanger un message "hello messages" pour se mettre d'accord sur l'algorithme.
\item Echanger des paramètres cryptographiques pour accepter le secret entre le client et le serveur.
\item Echanger les certificats et des informations cryptographiques pour permettre au client et au serveur de s'authentifier. 
\item Générer un secret à partir d'un autre et échange des valeurs aléatoires.
\item Fournir les paramètres de sécurité.
\item Permettre au client et au serveur de vérifier qu'ils ont le même paramètre de sécurité et que le handshake a eu lieu sans altération d'un attaquant.
\end{itemize}
\paragraph{Hello request}
C'est une notification pour que le client initie la négociation d'une connexion.
Le message "Hello request" peut être envoyé à n'importe quel moment.
\paragraph{Client Hello}
Lors de la première connexion du client au serveur, il est nécessaire d'envoyer un "Client hello" comme premier message.
Le client peut aussi envoyer un "Client hello" comme une réponse de "Hello request".
Ce message contient la date, un nonce et les algorithmes disponibles.
\paragraph{Server Hello}
Le serveur va envoyer ce message suite a un "Client hello" si il y a un algorithme commun, sinon il envoie un "Failure alert".
Ce message contient la date, un nonce et l'algorithme choisi.
\paragraph{Server Certificate}
Le serveur envoie son certificat pour s'authentifier auprès du client.
Ce message contient un Site-Cert ainsi que les certificats de la chaîne de certification (Autorité).
\paragraph{Client Certificate Request}
Optionnel, seulement si le serveur veut que le client soit authentifié.
\paragraph{Server Hello Done}
Indique la fin d'envoi du serveur.
\paragraph{Client Certificate}
Si (Client Certificate Request) est émis, alors le client envoie son certificat pour l'authentification client. 
\paragraph{Client Key Exchange}
Paquet chiffré avec la clé publique du serveur qui contient une clé de session générée à partir des deux nonces échangés. Si le serveur est capable de déchiffrer et de répondre, il est authentifié auprès du client.
\paragraph{Client Verify}
Si (Client Certificate Request) est émis, le client devra signer avec sa clé privée un haché des échanges précédents, ce qui l'authentifiera auprès du serveur. 
\paragraph{Change Cipher Spec.}
Précise que tous les paquets envoyés à la suite du "Client Finished" seront chiffrés avec la clé de session échangée et les algorithmes choisis. 
\paragraph{Client Finished}
Informe que le client a fini et contient un haché de la totalité des échanges. 
\paragraph{Change Cipher Spec.}
Précise qu'à partir de maintenant, le serveur va envoyer des paquets chiffrés. 
\paragraph{Server Finished}
Contient un haché de tous les échanges chiffré avec la clé de session et un MAC.

\section{Conception}

\subsection{UML}
\includegraphics[width=0.8\textwidth]{images/uml.pdf}
~~\\
~~\\
Rôle des différentes classes:
\begin{itemize}
	\item Transchiffrement: classe principale de l'application, elle contient le main et va permettre la création de la socket serveur et la détection du type de connexion entrantes (HTTP ou HTTPS).
	\item GenerationCertificat: cette classe permet de forger un faux certificat en fonction de celui récupéré lors de l'établissement d'une connexion HTTPS.
	\item Transfert: classe permettant la création de threads pour l'échange des données entre les différentes entités (voir schéma partie 2.4.1).
	\item Connexion: une classe abstraite qui permet de mutualiser le code commun entre les connexions HTTP ou HTTPS.
	\item ConnexionHTTP: permet de gérer les connexions HTTP, avec le lancement de deux threads pour l'échange des données grâce a la classe Transfert.
	\item ConnexionHTTPS: permet de gérer les connexions HTTPS, appel de la classe GenerationCertificat pour forger le faux certificat et lancement de deux threads pour l'échange des données grâce a la classe Transfert.
	\item Constantes: cette classe regroupe toutes les valeurs constantes utilisées dans la plupart des classes du projet.
	\item JournalFichier: cette classe permet la création, l'ouverture et le remplissage d'un fichier de avec les logs récupérés lors des échanges de type HTTP et HTTPS.
\end{itemize}

\section{Implémentation}
\subsection{Faire accepter une autorité}
De nos jours, nous avons besoin d'authentifier les sites auxquels on veut accéder pour être sur que les données cruciales que nous manipulons ne tombent pas en de mauvaises mains. Pour ce faire, on utilise des certificats signés par des autorités en qui on peut avoir confiance.
Ces autorités sont répertoriées dans nos machines et plus précisément dans nos navigateurs par le biais de certificats d'autorité.
Nous allons maintenant voir comment on peut faire accepter l'installation d'une nouvelle autorité sur sa machine en quelques clics.




\subsection{Installation par un administrateur mal intentionné}
Nous prenons, ici, le cas d'un administrateur ayant un accès à tous les ordinateurs d'une entreprise.
Cette personne veut faire accepter une autorité de certification dont il est le propriétaire pour pouvoir lire tous les paquets qui transitent et surtout les chiffrés.
Nous allons expliquer comment faire pour les principaux navigateurs utilisés sous linux.
Voilà un tableau comparatif des pourcentages d'utilisation des navigateurs : 

Moyenne générale (Janvier 2014) :
\begin{itemize}
\item{Chrome} 		43.67\%
\item{Internet Explorer}		 22,85\%
\item{Firefox} 		18,9\%
\item{Safari} 		9.73\%
\item{Opera}	 		1,3\%	
\item{Autres} 		3.55\%
\end{itemize}	


D'après le site : \begin{verbatim}
http://gs.statcounter.com/
\end{verbatim}

Nous allons privilégier Firefox et Chrome, cela nous permettra de viser plus de 60\% des machines.

Nous pouvons imaginer que l'administrateur installe ces navigateurs sur les machines de l'entreprise.
\newpage



\subsection{Installation par l'utilisateur}
Dans ce cas, peu importe le navigateur du client, on lui demande d'installer un certificat sur sa machine.

Lorsqu'un utilisateur


\subsection{Collision MD5}
Dans ce cas, aucune opération manuelle n'est nécessaire, ni par l'utilisateur, ni par l'administrateur.




\subsection{Threads}

Une fois une connexion établie, nous avons au niveau du proxy deux Sockets (bi-directionnelles), une vers le serveur, et une vers le client.

Chaque Socket est composée de deux Stream (uni-directionnels), un en entrée et l'autre en sortie.

Jusqu'à ce que la connexion soit interrompue, nous devons faire transiter les paquets, de l'entrée vers la sortie, et réciproquement pour le deuxième stream de la socket.

Pour ce faire, pour chaque nous lançons un objet de type Transfert dans un nouveau Thread.

Cela permet de continuer les traitements en parallèle, sans que l'application soit bloquée par un read.

\includegraphics[width=0.8\textwidth]{images/thread.pdf}

\subsection{Sockets}
Pour ce projet, nous devons faire communiquer les différents composants du projet ensemble. Ces composants sont :
\begin{itemize}
\item le navigateur du client ;
\item le proxy ;
\item le serveur Web.
\end{itemize}

Suite à la phase d'analyse, nous avons décidé d'utiliser les sockets, qui sont implémentées en Java.

\subsection{Keystore}
En Java, pour manipuler des certificats, nous utilisons des objets de types keyStore.

Le keyStore doit contenir la clé privée et le certificat de l'application, dans notre cas, le proxy.

Dans notre cas, nous avons aussi besoin de reconnaître l'autorité qui a signé le certificat du site Web distant.

Ensuite, à chaque fois que l'utilisateur demande une connexion vers un site Web, il faut générer un certificat Fake-Cert. Ce certificat reprend les différentes informations du certificat authentique, à l'exception de la clé publique.

L'intérêt de la manipulation est de posséder la clé privée associée.

javax.net.ssl.keyStorePassword - Password to access the private key from the keystore file specified by javax.net.ssl.keyStore. This password is used twice: To unlock the keystore file (store password), and To decrypt the private key stored in the keystore (key password).

javax.net.ssl.trustStore - Location of the Java keystore file containing the collection of CA certificates trusted by this application process (trust store). On Windows, the specified pathname must use forward slashes, /, in place of backslashes.

Si nous ne définissons pas de keyStore, il en existe un par défaut, à l'emplacement :
\begin{verbatim}
$JAVA_HOME/lib/security/jssecacerts
$JAVA_HOME/lib/security/cacerts
\end{verbatim}


A truststore contains CA certifcates to trust. If your server’s certificate is signed by a recognized CA, the default truststore that ships with the JRE will already trust it (because it already trusts trustworthy CAs), so you don’t need to build your own, or to add anything to the one from the JRE.







In SSL handshake purpose of trustStore is to verify credentials and purpose of keyStore is to provide credential.

keyStore in Java stores private key and certificates corresponding to there public keys and require if you are SSL Server or SSL requires client authentication.

TrustStore stores certificates from third party, your Java application communicate or certificates signed by CA(certificate authorities like Verisign, Thawte, Geotrust or GoDaddy) which can be used to identify third party.

TrustManager determines whether remote connection should be trusted or not i.e. whether remote party is who it claims to and KeyManager decides which authentication credentials should be sent to the remote host for authentication during SSL handshake.

If you are an SSL Server you will use private key during key exchange algorithm and send certificates corresponding to your public keys to client, this certificate is acquired from keyStore. On SSL client side, if its written in Java, it will use certificates stored in trustStore to verify identity of Server. SSL certificates are most commonly comes as .cer file which is added into keyStore or trustStore by using any key management utility e.g. keytool.

\subsection{Journalisation des échanges}
Un des objectif du proxy de transchiffrement est de pouvoir enregistrer tous les 
échanges entre un client et un serveur lors de leur communication. Dans ce but nous
avons réalisé une classe qui nous permet de stocker dans un fichier tout le trafic qui traverse le proxy.

Ensuite, nous avons développé une seconde méthode, pour chercher un éventuel champ password
dans un formulaire POST, ou passé avec GET. Même si en théorie le mot de passe ne doit jamais
être passé dans l'adresse, il existe toujours des développeurs peu précautionneux.
\section{Abstract}
\subsection{Authentification client}

L'authentification client sur un site web avec l'utilisation de certificats permet de créer une authentification forte.
Ce type d'authentification est beaucoup plus sûr qu'une authentification par 
login et mot de passe, trop facilement trouvable par un attaquant.

Cependant, il est rare qu'un site web sécurisé avec HTTPS utilise de l'authentification 
client. Pour gérer une authentification client, les serveurs doivent être configurés d'une certaine manière. 
Pour exemple le site des impôts Français a essayer de mettre en place ce 
type d'authentification mais cela c'est révélé être un échec dû à la difficulté 
apparente pour une majeur partie des utilisateurs.

Dans le cadre de notre projet, l'utilisation d'une authentification client entre le proxy et le serveur web n'est 
pas réalisable du fait que nous ne gérons pas les serveurs des sites web, et il 
est impossible d'imposer à un site une authentification client alors qu'il 
n'implémente pas cette méthode.

D'autre part si le site web demandé par un client demande l'authentification de 
ce dernier, elle ne sera pas possible à implémentée. Un certificat client est 
généré et signé par l'AC du serveur, or le proxy établit une connexion SSL avec 
le client en utilisant ca propre AC, et ne pourra donc pas reconnaitre le 
certificat du client comme valide, lors de la création du contexte SSL avec le client. 



\section{Tests}

\chapter{Recherche de collision sur des certificats hachés en MD5}

\chapter{Synthèse}

\chapter{Annexes}
\subsection{Mozilla Firefox}

Tout d'abord, l'administrateur ouvre Firefox puis clique sur Edit > Preferences.

\includegraphics[width=\textwidth]{images_autorites/OngletPref.png}
\newpage
Ensuite, il choisit Advanced > Certificates > View Certificates.

\includegraphics[width=\textwidth]{images_autorites/OngletCert.png}
\newpage
Il va ensuite dans Authorities > Import.

\includegraphics[width=\textwidth]{images_autorites/OngletCA.png}
\newpage
Il choisit ensuite le certificat de l'autorité qu'il veut installer puis valide.

\includegraphics[width=\textwidth]{images_autorites/OngletImport.png}
\newpage
Une fenêtre s'ouvre et propose de faire confiance à cette autorité pour 3 types de Certificats. L'administrateur coche les 3 cases pour que son autorité soit reconnue valide sur tous les types puis clique sur ok.

\includegraphics[width=\textwidth]{images_autorites/OngletConfirm.png} 


Voilà, l'autorité est installée et tous les certificats signés par cette autorité seront reconnus comme valides.
\newpage
\subsection{Chrome}

La démarche est très similaire à celle de firefox.

Tout d'abord, l'administrateur va dans Modifier > Préférences

\includegraphics[width=\textwidth]{images_autorites/ChromePref.png} 
\newpage

Puis il clique sur Afficher les paramètres avancés.

\includegraphics[width=\textwidth]{images_autorites/ChromeAvance.png} 
\newpage

Ensuite, dans la partie HTTPS/SSL, il clique sur Gérer les certificats

\includegraphics[width=\textwidth]{images_autorites/ChromeCert.png} 
\newpage

Il se déplace dans Autorités et clique sur Importer

\includegraphics[width=\textwidth]{images_autorites/ChromeCA.png} 
\newpage

Il choisit le certificat de l'autorité et valide.

\includegraphics[width=\textwidth]{images_autorites/ChromeImport.png} 
\newpage

Enfin, il coche les 3 cases et clique sur ok pour finaliser l'installation de l'autorité.

\includegraphics[width=\textwidth]{images_autorites/ChromeValide.png} 
\newpage

\section{Forcer l'acceptation de l'autorité par un client}
Dans cette section, nous forçons l'utilisateur a accepter notre autorité de certification. Si il ne l'a pas, il ne pourra pas naviguer sur internet.
Nous proposons donc un lien pour récupérer le certificat d'autorité sur lequel il faut cliquer.

\includegraphics[width=\textwidth]{images_autorites/Page.png} 
\newpage

Ensuite, la fenêtre de validation s'ouvre et le client doit cocher les cases puis valider.

\includegraphics[width=\textwidth]{images_autorites/Cert.png} 
\newpage

A ce stade, l'autorité est installée et si le client veut retenter de l'installer, un message le prévient qu'il a déjà fini l'installation.

\includegraphics[width=\textwidth]{images_autorites/Alerte.png}

On peut voir qu'en seulement quelques clics, l'utilisateur installe une autorité dont il ne connaît rien et qui peut être utilisée pour déchiffrer toutes ses informations personnelles. 
\section{Captures d'écran de l'installation de l'autorité}

\end{document}
