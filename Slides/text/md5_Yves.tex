\begin{frame}
		\frametitle{Attaque de Wang}		
		\begin{block}{Type d'attaque}
		\begin{itemize}
		\item attaque par collision
		  \begin{itemize}
		  \item trouver M = ($M_0, M_1$) et M' = ($M'_0, M'_1$) tel que MD5(M) = MD5(M')
		  \end{itemize}
		\item Utiliser une m\'ethode dite diff\'erentielle
		 \begin{itemize}
		 \item analyser les diff\'erences fix\'ee entre deux messages 
		 \end{itemize}
		\end{itemize}
		\end{block}
		
		\begin{block}{Caract\'erisation}
		\begin{itemize}
		\item (a, b, c, d) = MD5($a_0, b_0, c_0, d_0, M_0$)
		\item (a', b', c', d') = MD5($a_0, b_0, c_0, d_0, M'_0$)
		\item on a MD5($a, b, c, d, M_1$) = MD5($a', b', c', d', M'_1$).
\end{itemize}
		\end{block}
	\end{frame}

	\begin{frame}
		\frametitle{D\'eroulement de l'attaque}
		\begin{block}{Condition pour mener l'attaque}
		  \begin{enumerate}
		  \item Diff\'erence entre les premiers blocs de 512 bits 
		    \begin{itemize}
		    \item $\Delta$$M_{0}$ = $M'_{0}$ - $M_{0}$ = (0, 0, 0, 0, $2^{31}$ , 0, 0, 0, 0, 0, 0, $2^{15}$, 0, 0, $2^{31}$, 0)
		    \end{itemize}
%		  \item Diff\'erence entre les seconds blocs de 512 bits 
		    \begin{itemize}
		    \item $\Delta$$M_{0}$ = $M'_{0}$ - $M_{0}$ = (0, 0, 0, 0, $2^{31}$, 0, 0, 0, 0, 0, 0, -$2^{15}$, 0, 0, $2^{31}$, 0)
		    \end{itemize}
		  \item Diff\'erence entre les signatures des deux premiers blocs
		   \begin{itemize}
		   \item $\Delta$$IV_{1}$ = $IV'_{1}$ - $IV_{1}$ = ($2^{31}$, $2^{31}$ + $2^{31}$, $2^{15}$ + $2^{31}$, $2^{15}$ + $2^{31}$)
		   \end{itemize}

		  \end{enumerate}
		\end{block}
	
		\begin{block}{R\'esum\'e}
		 \begin{itemize}
		 \item Ne permet pas le calcul de pr\'e-image.
		 \item M et M' sont deux messages de 1024 bits chacun.
		 \item M et M' sont obtenus apr\`es deux passages dans la fonction MD5Compress.
		 \item Complexité de recherche
		 \begin{itemize}
		 \item premier bloc: $2^{33}$
		 \item second bloc: $2^{29}$
		 \end{itemize}
		\end{itemize}
		\end{block}
	\end{frame} 




	\begin{frame}
		\frametitle{Attaque de Marc Stevens}
		\begin{block}{}
		  \begin{itemize}
		  \item S'appuie sur les recherches de Wang.
		  \item Permet le calcul de pr\'efixe.
		  \item Construction progressive des blocs de collisions.
		  \end{itemize}
		\end{block}
	
		\begin{block}{Caract\'erisation}
		  \begin{itemize}
		  \item Trouver M = P . S et  M' = P' . S' tel que MD5(M) = MD5(M').
		  \item (P, P') pr\'efixes de (M, M') dont les valeurs sont fixes.
		  \item (S, S') suffixes de (M, M')
		  \end{itemize}
		\end{block}
	\end{frame}

	
	\begin{frame}
		\frametitle{Attaque de Marc Stevens}
		\begin{block}{D\'ecomposition des suffixes S et S'}
		S et S' ont la m\^eme structure qui suit:
		  \begin{itemize}
		  \item $S_r$: padding choisit tel que P . $S_r$ . $S_b$ . $S_c$ multiple de 512.
		  \item $S_b$: bits d'anniversaires.
		  \item $S_c$: bits de quasi collisions.
		  \end{itemize}
		\end{block}
		
		\begin{block}{Condition pour mener l'attaque}
		  \begin{itemize}
		  \item Utilise la m\'ethode diff\'erentielle et des bits de conditions.
		    \begin{itemize}
		    \item Permet de calculer un chemin diff\'erentiel \`a partir de $\Delta$IHV et $\Delta$B.
		    \item Le chemin est d\'efini \`a l'aide de bits de conditions.
		    \end{itemize}
		  \end{itemize}
		  
		  La complexité de cette m\'ethode est de $2^{16}$ appels \`a la fonction de compression MD5.
		\end{block}
	\end{frame}
	
			
			
			
			
	\begin{frame}
		\frametitle{Application aux certificats}
		\begin{block}{Points essentiels}
		  \begin{itemize}
		  \item D\'etermination du pr\'efixe.
		  \item O\`u op\'erer la recherche des blocs de collisions.
		  \item Propagation des blocs de collisions.
		  \end{itemize}
		\end{block}
		
		\begin{block}{Principe}
		  \begin{itemize}
		  \item G\'en\'erer deux certificats \`a faire signer par l'autorit\'e: un vrai et un faux
		    \begin{itemize}
		    \item D\'eterminer le d\'ebut des modules RSA.
		    \item Modification du module RSA du faux pour en contr\^oler la cl\'e priv\'ee.
		    \end{itemize}
		  \end{itemize}
		\end{block}
		
	\frame{
	\frametitle{Exemple de certificats}
	
	}
	\end{frame}
