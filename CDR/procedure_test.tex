\documentclass[a4paper,11pt,french]{article}
\usepackage[utf8]{inputenc}

\usepackage[T1]{fontenc}
\usepackage[francais]{babel} 
\usepackage[top=2cm, bottom=2cm, left=2cm, right=2cm, includeheadfoot]{geometry} %pour les marges
\usepackage{lmodern}
\usepackage{pict2e}
\usepackage{tikz}	
\usepackage{tikz-uml}
\usepackage{fancyhdr} % Required for custom headers
\usepackage{lastpage} % Required to determine the last page for the footer
\usepackage{extramarks} % Required for headers and footers
\usepackage{graphicx} % Required to insert images
\usepackage{tabularx, longtable}
\usepackage{color, colortbl}
\usepackage{lscape}
%\usepackage[hidelinks]{hyperref}
\usepackage{longtable}
\usepackage{multirow}
\usepackage{rotating}
%\usepackage{pgfgantt}
%\usepackage{pgfcalendar}
%\usepackage{ifthen}
\usepackage{gensymb}
\definecolor{Blue}{HTML}{6495ED}
\definecolor{Orange}{HTML}{FBCEB1}

\linespread{1.1} % Line spacing

% Set up the header and footer
\pagestyle{fancy}
\lhead{\textbf{\hmwkClass -- \hmwkSubject \\ \hmwkTitle \\ \hmwkDocName}} % Top left header
\rhead{\includegraphics[width=10em]{logo_univ.png}}
\lfoot{\lastxmark} % Bottom left footer
\cfoot{} % Bottom center footer
\rfoot{Page\ \thepage\ / \pageref{LastPage}} % Bottom right footer
\renewcommand\headrulewidth{0.4pt} % Size of the header rule
\renewcommand\footrulewidth{0.4pt} % Size of the footer rule

\setlength{\headheight}{40pt}

\newcommand{\hmwkTitle}{Transchiffrement} % Assignment title
\newcommand{\hmwkClass}{Master 2 SSI } % Course/class
\newcommand{\hmwkAuthorName}{Jean-Baptiste Souchal} % Your name
\newcommand{\hmwkSubject}{Conduite de projet} % Subject
\newcommand{\hmwkDocName}{Procédures de tests} % Document name

\newcommand{\version}{1.0} % Document version
\newcommand{\docDate}{20 Janvier 2014} % Document date
\newcommand{\checked}{} % Checker name
\newcommand{\approved}{} % Approver name

\makeatletter
\newcommand{\resettranslate}{\let\translate\@firstofone}
\makeatother

\definecolor{gris}{rgb}{0.95, 0.95, 0.95}

\title{
\vspace{2in}
\textmd{\textbf{\hmwkClass :\ \hmwkTitle}}\\
\normalsize\vspace{0.1in}\small{Due\ on\ \hmwkDueDate}\\
\vspace{0.1in}\large{\textit{\hmwkClassInstructor\ \hmwkClassTime}}
\vspace{3in}
}

\author{\hmwkAuthorName}
\date{} % Insert date here if you want it to appear below your name


\usepackage{amsmath}
\begin{document}
\newcount\startdate
\newcount\daynum
%\pgfcalendardatetojulian{2013-01-021}{\startdate}
\pagestyle{fancy}

\vspace*{5cm}
\begin{center}\textbf{\Huge{\hmwkDocName}}\end{center}
\vspace*{4.5cm}
	

\fcolorbox{black}{gris}{
\begin{minipage}{15cm}
\begin{tabularx}{10cm}{lXl}
	\bfseries{Version} & & \version\\
	& & \\
	\bfseries{Date} & & \docDate\\
	& & \\
	\bfseries{Rédigé par} & & \hmwkAuthorName \\
	& & \\
	\bfseries{Relu par} & & \checked \\
	& & \\
	\bfseries{Approuvé par} & & \approved \\
	& & \\
\end{tabularx}
\end{minipage}
}

\newpage

%Tableau de mises à jour
\vspace*{1cm}
\begin{center}
\textbf{\huge{MISES À JOUR}}\\
\vspace*{3cm}
	\begin{tabularx}{16cm}{|c|c|X|}
	\hline
	\bfseries{Version} & \bfseries{Date} & \bfseries{Modifications réalisées}\\
	\hline
	0.1 & 20/01/2014 & Création de la version \LaTeX\\
	\hline
	\end{tabularx}
\end{center}

%La table des matières
\clearpage
\tableofcontents
\clearpage
\newpage

\section{Cas de tests}
\subsection{Installation de l'application proxy}

\begin{tabular}{|m{2.5cm}|m{4cm}|m{3cm}|m{3.5cm}|m{2cm}|}
\hline 
\rowcolor{Blue} Objet & \multicolumn{4}{|l|}{Installation de l'application Proxy} \\ 
\hline 
\rowcolor{Blue} Objectif & \multicolumn{4}{|l|}{Vérifier l'installation de l'application Proxy} \\ 
\hline 
\rowcolor{Blue} Pré-requis & \multicolumn{4}{|l|}{Java Runtime Environnement installé sur la machine} \\ 
\hline 
\rowcolor{Orange} ID & Actions & Données & Résultats attendus & OK/KO \\ 
\hline 
1 & On clique sur l'exécutable de l'application & exécutable proxy & l'application s'installe sur le client &  \\ 
\hline
2 & On vérifie que le processus associé au proxy est actif sur la machine & & un processus associé au proxy est actif sur le client & \\
\hline
Commentaires & \multicolumn{4}{|l|}{} \\ 
\hline
\end{tabular}

\newpage

\subsection{Installation de l'autorité}

\begin{tabular}{|m{2.5cm}|m{4cm}|m{3cm}|m{3.5cm}|m{2cm}|}
\hline 
\rowcolor{Blue} Objet & \multicolumn{4}{|l|}{Installation de l'autorité} \\ 
\hline 
\rowcolor{Blue} Objectif & \multicolumn{4}{|l|}{Vérifier l'installation de l'autorité sur la machine utilisateur} \\ 
\hline 
\rowcolor{Blue} Pré-requis & \multicolumn{4}{|l|}{Pas d'autorité présente sur la machine, navigateur qui accepte les certificats MD5} \\ 
\hline 
\rowcolor{Orange} ID & Actions & Données & Résultats attendus & OK/KO \\ 
\hline 
1 & Installer l'autorité directement sur le navigateur du client & certificat de l'autorité & le navigateur accepte le certificat &  \\ 
\hline 
Commentaires & \multicolumn{4}{|l|}{} \\ 
\hline
\end{tabular}

\newpage

\subsection{Acceptation de l'autorité}

\begin{tabular}{|m{2.5cm}|m{4cm}|m{3cm}|m{3.5cm}|m{2cm}|}
\hline 
\rowcolor{Blue} Objet & \multicolumn{4}{|l|}{Acceptation de l'autorité} \\ 
\hline 
\rowcolor{Blue} Objectif & \multicolumn{4}{|l|}{Vérifier l'installation de l'autorité sur la machine utilisateur} \\ 
\hline 
\rowcolor{Blue} Pré-requis & \multicolumn{4}{|l|}{Pas d'autorité présente sur la machine} \\ 
\hline 
\rowcolor{Orange} ID & Actions & Données & Résultats attendus & OK/KO \\ 
\hline 
1 & Inclure le certificat dans l'acceptation des termes et conditions du proxy & certificat de l'autorité & le certificat de l'autorité est inclus dans l'acceptation du proxy &  \\ 
\hline
2 & Le client accepte les termes et conditions du proxy & & le certificat de 
l'autorité va automatiquement s'installer sur le navigateur du client & 
\\
\hline 
Commentaires & \multicolumn{4}{|l|}{} \\ 
\hline
\end{tabular}

\newpage

\subsection{Transfert des paquets HTTP}

\begin{tabular}{|m{2.5cm}|m{4cm}|m{3cm}|m{3.5cm}|m{2cm}|}
\hline 
\rowcolor{Blue} Objet & \multicolumn{4}{|l|}{Transfert des paquets HTTP en clair} \\ 
\hline 
\rowcolor{Blue} Objectif & \multicolumn{4}{|l|}{Assurer le transfert des paquets HTTP du client vers le serveur} \\ 
\hline 
\rowcolor{Blue} Pré-requis & \multicolumn{4}{|l|}{Proxy configuré pour le mode HTTP} \\ 
\hline 
\rowcolor{Orange} ID & Actions & Données & Résultats attendus & OK/KO \\ 
\hline 
1 & le client envois une requête HTTP sur un serveur &  & le proxy reçoit la requête HTTP &  \\ 
\hline
2 & le proxy envois la requête HTTP vers le serveur & & le serveur reçoit la requête HTTP & \\
\hline
3 & le serveur envois la réponse au proxy & & le proxy 
reçoit la réponse du serveur & \\
\hline 
4 & le proxy envois la réponse vers le client & & le client reçoit la réponse & 
\\
\hline
Commentaires & \multicolumn{4}{|l|}{} \\ 
\hline
\end{tabular}

\newpage

\subsection{Etablissement d'une session privée}

\begin{tabular}{|m{2.5cm}|m{4cm}|m{3cm}|m{3.5cm}|m{2cm}|}
\hline 
\rowcolor{Blue} Objet & \multicolumn{4}{|l|}{Création des sessions privées entre client-proxy-serveur} \\ 
\hline 
\rowcolor{Blue} Objectif & \multicolumn{4}{|l|}{Détailler les étapes d'ouverture des sessions privées} \\ 
\hline 
\rowcolor{Blue} Pré-requis & \multicolumn{4}{|l|}{deux connexions SSL/TLS ouvertes, certificat de l'autorité installé sur le client} \\ 
\rowcolor{Blue} & \multicolumn{4}{|l|}{et proxy fonctionnel} \\
\hline
\rowcolor{Orange} ID & Actions & Données & Résultats attendus & OK/KO \\ 
\hline 
1 & le client envois une requête HTTPS & URL du serveur & le proxy reçoit la requête HTTPS &  \\ 
\hline
2 & le proxy envois la requête HTTPS vers le serveur & & le serveur reçoit la requête HTTPS & \\
\hline
3 & le serveur envois son certificat au proxy & certificat serveur & le proxy reçoit le certificat & \\
\hline 
4 & le proxy génère son certificat & certificat proxy & le proxy envois son certificat au client & \\
\hline
5 & le client reçoit le certificat du proxy & idem & le client vérfie la 
validité du certificat & \\
\hline
6 & le client accepte le certificat & & le client envoi une clé de session au 
proxy & \\
\hline
7 & le proxy reçoit la clé de session & clé session client & le proxy génère sa clé de session & \\
\hline
8 & le proxy envois sa clé de session au serveur & clé session proxy & le serveur 
reçoit la clé de session & \\
\hline
9 & session ouverte entre le serveur et le proxy & & le serveur et le proxy 
peuvent échanger des messages chiffrés & \\
\hline
10 & session ouverte entre le proxy et le client & & le proxy et le client 
peuvent échanger des messages chiffrés & \\
\hline
Commentaires & \multicolumn{4}{|l|}{} \\ 
\hline
\end{tabular}

\newpage


\subsection{Transchiffrement des échanges HTTPS}

\begin{tabular}{|m{2.5cm}|m{4cm}|m{3cm}|m{3.5cm}|m{2cm}|}
\hline 
\rowcolor{Blue} Objet & \multicolumn{4}{|l|}{Transfchiffrement échanges HTTPS} \\ 
\hline 
\rowcolor{Blue} Objectif & \multicolumn{4}{|l|}{Transchiffrement sur les connexions HTTPS de manière transparente} \\
\rowcolor{Blue} & \multicolumn{4}{|l|}{pour l'utilisateur} \\ 
\hline 
\rowcolor{Blue} Pré-requis & \multicolumn{4}{|l|}{Session privée établie, Certificat généré, deux connexions SSL/TLS ouvertes,} \\ 
\rowcolor{Blue} & \multicolumn{4}{|l|}{certificat de l'autorité installé sur le client et proxy fonctionnel} 
\\
\hline 
\rowcolor{Orange} ID & Actions & Données & Résultats attendus & OK/KO \\ 
\hline 
1 & le client envois une requête HTTPS au proxy & URL du serveur & le proxy reçoit la requête HTTPS &  \\ 
\hline
2 & le proxy déchiffre la requête HTTPS & & le proxy log la requête en clair & \\
\hline
3 & le proxy chiffre la requête & clé session proxy-serveur & la requête est 
correctement chiffrée & \\
\hline
4 & le proxy envois la requête chiffrée au serveur & & le serveur reçoit la requête & \\
\hline
5 & le envois au proxy & & le proxy reçois la 
requête & \\
\hline
6 & le proxy déchiffre la réponse du serveur & & le proxy log la réponse au 
serveur en clair & \\
\hline
7 & le proxy chiffre la réponse & clé session client-proxy & la réponse est 
correctement chiffré & \\
\hline
8 & le proxy envois la réponse chiffrée au client & & le client reçoit la réponse 
& \\
\hline
9 & le client déchiffre la réponse & & le client peut lire et traiter le réponse 
& \\
\hline
Commentaires & \multicolumn{4}{|l|}{} \\ 
\hline
\end{tabular}

\newpage

\end{document}