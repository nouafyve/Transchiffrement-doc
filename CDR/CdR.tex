\documentclass[a4paper,11pt,french]{article}
\usepackage[utf8]{inputenc}

\usepackage[T1]{fontenc}
\usepackage[francais]{babel} 
\usepackage[top=2cm, bottom=2cm, left=2cm, right=2cm, includeheadfoot]{geometry} %pour les marges
\usepackage{lmodern}
\usepackage{pict2e}
\usepackage{tikz}	
\usepackage{tikz-uml}
\usepackage{fancyhdr} % Required for custom headers
\usepackage{lastpage} % Required to determine the last page for the footer
\usepackage{extramarks} % Required for headers and footers
\usepackage{graphicx} % Required to insert images
\usepackage{tabularx, longtable}
\usepackage{color, colortbl}
\usepackage{lscape}
%\usepackage[hidelinks]{hyperref}
\usepackage{longtable}
\usepackage{multirow}
\usepackage{rotating}
%\usepackage{pgfgantt}
%\usepackage{pgfcalendar}
%\usepackage{ifthen}
\usepackage{gensymb}


\linespread{1.1} % Line spacing

% Set up the header and footer
\pagestyle{fancy}
\lhead{\textbf{\hmwkClass -- \hmwkSubject \\ \hmwkTitle \\ \hmwkDocName}} % Top left header
\rhead{\includegraphics[width=10em]{logo_univ.png}}
\lfoot{\lastxmark} % Bottom left footer
\cfoot{} % Bottom center footer
\rfoot{Page\ \thepage\ / \pageref{LastPage}} % Bottom right footer
\renewcommand\headrulewidth{0.4pt} % Size of the header rule
\renewcommand\footrulewidth{0.4pt} % Size of the footer rule

\setlength{\headheight}{40pt}

\newcommand{\hmwkTitle}{Transchiffrement} % Assignment title
\newcommand{\hmwkClass}{Master 2 SSI } % Course/class
\newcommand{\hmwkAuthorName}{Jean-Baptiste Souchal} % Your name
\newcommand{\hmwkSubject}{Conduite de projet} % Subject
\newcommand{\hmwkDocName}{Cahier de Recette} % Document name

\newcommand{\version}{1.0} % Document version
\newcommand{\docDate}{20 Janvier 2014} % Document date
\newcommand{\checked}{} % Checker name
\newcommand{\approved}{} % Approver name

\makeatletter
\newcommand{\resettranslate}{\let\translate\@firstofone}
\makeatother

\definecolor{gris}{rgb}{0.95, 0.95, 0.95}

\title{
\vspace{2in}
\textmd{\textbf{\hmwkClass :\ \hmwkTitle}}\\
\normalsize\vspace{0.1in}\small{Due\ on\ \hmwkDueDate}\\
\vspace{0.1in}\large{\textit{\hmwkClassInstructor\ \hmwkClassTime}}
\vspace{3in}
}

\author{\hmwkAuthorName}
\date{} % Insert date here if you want it to appear below your name


\usepackage{amsmath}
\begin{document}
\newcount\startdate
\newcount\daynum
%\pgfcalendardatetojulian{2013-01-021}{\startdate}
\pagestyle{fancy}

\vspace*{5cm}
\begin{center}\textbf{\Huge{\hmwkDocName}}\end{center}
\vspace*{4.5cm}
	

\fcolorbox{black}{gris}{
\begin{minipage}{15cm}
\begin{tabularx}{10cm}{lXl}
	\bfseries{Version} & & \version\\
	& & \\
	\bfseries{Date} & & \docDate\\
	& & \\
	\bfseries{Rédigé par} & & \hmwkAuthorName \\
	& & \\
	\bfseries{Relu par} & & \checked \\
	& & \\
	\bfseries{Approuvé par} & & \approved \\
	& & \\
\end{tabularx}
\end{minipage}
}

\newpage

%Tableau de mises à jour
\vspace*{1cm}
\begin{center}
\textbf{\huge{MISES À JOUR}}\\
\vspace*{3cm}
	\begin{tabularx}{16cm}{|c|c|X|}
	\hline
	\bfseries{Version} & \bfseries{Date} & \bfseries{Modifications réalisées}\\
	\hline
	0.1 & 10/12/2013 & Création\\
	\hline
	1.0 & 20/01/2014 & Modification\\
	\hline
	\end{tabularx}
\end{center}

%La table des matières
\clearpage
\tableofcontents
\clearpage
\newpage
\section{Objet}

Ce document a pour but de présenter une série de scénario décrivant avec précision les
démarches et procédures à suivre dans le cadre du fonctionnement du « proxy » réalisant
du transchiffrement SSL. Les différents tests s’exécuteront dans un environnement physique,
c’est à dire sur une machine. Les tests auront pour but de mettre en œuvre tous les scénarios
possibles du fonctionnement de l’application « proxy » et de déterminer si chacun d’eux est fonctionnel.
Ce document sert de support aux développeurs pour évaluer si le produit répond à leurs attentes, il permet
également de montrer au client l’état, en terme de fonctionnalité, l’avancement du projet.

\section{Portée}
Le cahier de recette est destiné au chef de projet et au client du projet de
Transchiffrement SSL ainsi qu’aux membres de l’équipe de développement.


\section{Document applicables et de référence}

\begin{itemize}
  \item Spécification technique des besoins
  \item Document d'architecture logiciel
\end{itemize}

\section{Terminologie et sigles utilisés}

\begin{itemize}
  \item STB : Spécification Technique des Besoins
  \item DAL : Document d’Architecture Logicielle
  \item Proxy : Application jouant le rôle de proxy afin de réaliser du transchiffrement SSL
  \item Client : Machine jouant le rôle d'utilisateur sur le réseau
  \item Serveur : Machine hébergeant le proxy (machine virtuelle, voir section Environnement de test)
\end{itemize}

\section{Environnement de test}
Les tests seront effectués dans un environnement de type Unix (Ubuntu, Mac OS 
X) pour la partie utilisateur. Utilisation d'une machine virtuelle (Ubuntu) pour l'hébergement du 
proxy, des tests sont à prévoir directement sur cette machine virtuelle. Les 
machine utilisées pour les tests requièrent une connexion internet ainsi qu'un
navigateur web supportant l'usage de certificats de type MD5. Aucune restriction 
sur la configuration matérielle des machines.
L'analyse des échanges entre les différentes entités se feront à l'aide de 
l'outils Wireshark.
\begin{itemize}
\item une machine virtuelle « proxy » ou sera installé le proxy 
\item une machine « client » qui jouera le rôle d'utilisateur lambda sur le réseau
\end{itemize}



\section{Responsabilité}

\begin{itemize}
  \item Client : Magali Bardet
  \item Chef de projet : Emile Générat
  \item Testeur/Recetteur : Jean-Baptiste Souchal
  \item Relecture des cas de test par le chef de projet pour une validation avant 
  chaque phase de tests sur l'application.
  \item Relecture du cahier de recette après la réalisation des tests par le 
  client pour la validation de chaque modules de l'application
\end{itemize}

\section{Stratégie de tests}
Un test est validé lorsqu’il répond à l’exigence fonctionnelle à laquelle il est lié.
Un test non validé fera l’objet d’un retour vers le(s) développeur(s) du module concerné.
Chaque test non validé implique la correction, par le(s) développeur(s), du module concerné dans
un délai raisonnable en fonction du planning et du plan de développement mis en place par le chef de
projet. Après correction, le module sera de nouveau testé. Après avoir effectué tous les tests, les
résultats seront envoyés au chef de projet et au client pour leurs validation.

\section{Gestion des anomalies}
Lors des tests, les anomalies seront rédigées dans un tableur. Il contiendra :

\begin{itemize}
  \item le numéro du test
  \item le module concerné
  \item une description de l'erreur
  \item les développeurs impliqués
  \item l’état d’avancement de la correction
\end{itemize}

Le responsable du module concerné par l’anomalie sera chargé de la résoudre dans un
  délai raisonnable en fonction de la gravité de cette anomalie pour le fonctionnement global
  de l’application. Le délai de correction devra être inclus dans le planning du développeur en fonction
  du plan de développement émis par le chef de projet.

\section{Procédures de test}
voir document procedure\_tests.pdf en annexe


\end{document}