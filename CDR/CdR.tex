\documentclass[a4paper,11pt,french]{article}
\usepackage[utf8]{inputenc}

\usepackage[T1]{fontenc}
\usepackage[francais]{babel} 
\usepackage[top=2cm, bottom=2cm, left=2cm, right=2cm, includeheadfoot]{geometry} %pour les marges
\usepackage{lmodern}
\usepackage{pict2e}
\usepackage{tikz}	
\usepackage{fancyhdr} % Required for custom headers
\usepackage{lastpage} % Required to determine the last page for the footer
\usepackage{extramarks} % Required for headers and footers
\usepackage{graphicx} % Required to insert images
\usepackage{tabularx, longtable}
\usepackage{color, colortbl}
\usepackage{lscape}
%\usepackage[hidelinks]{hyperref}
\usepackage{longtable}
\usepackage{multirow}
\usepackage{rotating}
%\usepackage{pgfgantt}
%\usepackage{pgfcalendar}
%\usepackage{ifthen}
\usepackage{gensymb}
\definecolor{Blue}{HTML}{6495ED}
\definecolor{Orange}{HTML}{FBCEB1}


\linespread{1.1} % Line spacing

% Set up the header and footer
\pagestyle{fancy}
\lhead{\textbf{\hmwkClass -- \hmwkSubject \\ \hmwkTitle \\ \hmwkDocName}} % Top left header
\rhead{\includegraphics[width=10em]{logo_univ.png}}
\lfoot{\lastxmark} % Bottom left footer
\cfoot{} % Bottom center footer
\rfoot{Page\ \thepage\ / \pageref{LastPage}} % Bottom right footer
\renewcommand\headrulewidth{0.4pt} % Size of the header rule
\renewcommand\footrulewidth{0.4pt} % Size of the footer rule

\setlength{\headheight}{40pt}

\newcommand{\hmwkTitle}{Transchiffrement} % Assignment title
\newcommand{\hmwkClass}{Master 2 SSI } % Course/class
\newcommand{\hmwkAuthorName}{Jean-Baptiste Souchal} % Your name
\newcommand{\hmwkSubject}{Conduite de projet} % Subject
\newcommand{\hmwkDocName}{Cahier de Recette} % Document name

\newcommand{\version}{1.0} % Document version
\newcommand{\docDate}{20 Janvier 2014} % Document date
\newcommand{\checked}{} % Julien Bourdon
\newcommand{\approved}{} % Approver name

\makeatletter
\newcommand{\resettranslate}{\let\translate\@firstofone}
\makeatother

\definecolor{gris}{rgb}{0.95, 0.95, 0.95}

\title{
\vspace{2in}
\textmd{\textbf{\hmwkClass :\ \hmwkTitle}}\\
\normalsize\vspace{0.1in}\small{Due\ on\ \hmwkDueDate}\\
\vspace{0.1in}\large{\textit{\hmwkClassInstructor\ \hmwkClassTime}}
\vspace{3in}
}

\author{\hmwkAuthorName}
\date{} % Insert date here if you want it to appear below your name


\usepackage{amsmath}
\begin{document}
\newcount\startdate
\newcount\daynum
%\pgfcalendardatetojulian{2013-01-021}{\startdate}
\pagestyle{fancy}

\vspace*{5cm}
\begin{center}\textbf{\Huge{\hmwkDocName}}\end{center}
\vspace*{4.5cm}
	

\fcolorbox{black}{gris}{
\begin{minipage}{15cm}
\begin{tabularx}{10cm}{lXl}
	\bfseries{Version} & & \version\\
	& & \\
	\bfseries{Date} & & \docDate\\
	& & \\
	\bfseries{Rédigé par} & & \hmwkAuthorName \\
	& & \\
	\bfseries{Relu par} & & \checked \\
	& & \\
	\bfseries{Approuvé par} & & \approved \\
	& & \\
\end{tabularx}
\end{minipage}
}

\newpage

%Tableau de mises à jour
\vspace*{1cm}
\begin{center}
\textbf{\huge{MISES À JOUR}}\\
\vspace*{3cm}
	\begin{tabularx}{16cm}{|c|c|X|}
	\hline
	\bfseries{Version} & \bfseries{Date} & \bfseries{Modifications réalisées}\\
	\hline
	0.1 & 10/12/2013 & Création\\
	\hline
	1.0 & 20/01/2014 & Prise en compte des modifications\\
	\hline
	\end{tabularx}
\end{center}

%La table des matières
\clearpage
\tableofcontents
\clearpage
\newpage
\section{Objet}

Ce document a pour but de présenter une série de scénarios décrivant avec précision les
démarches et procédures à suivre dans le cadre du fonctionnement du « proxy » réalisant
du transchiffrement SSL. Les différents tests s’exécuteront dans un environnement physique,
c’est à dire sur une machine. Les tests auront pour but de mettre en œuvre tous les scénarios
possibles du fonctionnement de l’application « proxy » et de déterminer si chacun d’eux est fonctionnel.
Ce document sert de support aux développeurs pour évaluer si le produit répond à leurs attentes, il permet
également de montrer au client l’état, en terme de fonctionnalité, de l’avancement du projet.

\section{Portée}
Le cahier de recette est destiné au chef de projet et au client du projet de
Transchiffrement SSL ainsi qu’aux membres de l’équipe de développement.


\section{Documents applicables et de référence}

\begin{itemize}
  \item Spécification technique des besoins
  \item Document d'architecture logiciel
\end{itemize}

\section{Terminologie et sigles utilisés}

\begin{itemize}
  \item STB : Spécification Technique du Besoin
  \item DAL : Document d’Architecture Logicielle
  \item Proxy : Application jouant le rôle de proxy afin de réaliser du transchiffrement SSL
  \item Client : Machine jouant le rôle d'utilisateur sur le réseau
  \item Serveur : Serveur web que le client tente de joindre
\end{itemize}

\section{Environnement de test}
Les tests seront effectués dans un environnement de type Unix (Ubuntu, Mac OS 
X) pour la partie utilisateur. Utilisation d'une machine virtuelle (Ubuntu) pour l'hébergement du 
proxy, des tests sont à prévoir directement sur cette machine virtuelle. Les 
machine utilisées pour les tests requièrent une connexion internet ainsi qu'un
navigateur web supportant l'usage de certificats de type MD5. Aucune restriction 
sur la configuration matérielle des machines.
L'analyse des échanges entre les différentes entités se fera à l'aide de 
l'outil Wireshark.
\begin{itemize}
\item une machine virtuelle « proxy » où sera installé le proxy 
\item une machine « client » qui jouera le rôle d'utilisateur lambda sur le réseau
\end{itemize}



\section{Responsabilité}

\begin{itemize}
  \item Client : Magali Bardet
  \item Chef de projet : Emile Générat
  \item Testeur/Recetteur : Jean-Baptiste Souchal
  \item Relecture des cas de test par le chef de projet pour une validation avant 
  chaque phase de tests sur l'application.
  \item Relecture du cahier de recette après la réalisation des tests par le 
  client pour la validation de chaque module de l'application
\end{itemize}

\section{Stratégie de tests}
Un test est validé lorsqu’il répond à l’exigence fonctionnelle à laquelle il est lié.
Un test non validé fera l’objet d’un retour vers le(s) développeur(s) du module concerné.
Chaque test non validé implique la correction, par le(s) développeur(s), du module concerné dans
un délai raisonnable en fonction du planning et du plan de développement mis en place par le chef de
projet. Après correction, le module sera de nouveau testé. Après avoir effectué tous les tests, les
résultats seront envoyés au chef de projet et au client pour leurs validation.

\section{Gestion des anomalies}
Lors des tests, les anomalies seront rédigées dans un tableur. Il contiendra :

\begin{itemize}
  \item le numéro du test
  \item le module concerné
  \item une description de l'erreur
  \item les développeurs impliqués
  \item l’état d’avancement de la correction
\end{itemize}

Le responsable du module concerné par l’anomalie sera chargé de la résoudre dans un
  délai raisonnable en fonction de la gravité de cette anomalie pour le fonctionnement global
  de l’application. Le délai de correction devra être inclus dans le planning du développeur en fonction
  du plan de développement émis par le chef de projet.
\newpage
\section{Procédures de test}
\subsection{Installation de l'application proxy}

\begin{tabular}{|m{2.5cm}|m{4cm}|m{3cm}|m{3.5cm}|m{2cm}|}
\hline 
\rowcolor{Blue} Objet & \multicolumn{4}{|l|}{Installation de l'application Proxy} \\ 
\hline 
\rowcolor{Blue} Objectif & \multicolumn{4}{|l|}{Vérifier l'installation de l'application Proxy} \\ 
\hline 
\rowcolor{Blue} Pré-requis & \multicolumn{4}{|l|}{Java Runtime Environnement installé sur la machine} \\ 
\hline 
\rowcolor{Orange} ID & Actions & Données & Résultats attendus & OK/KO \\ 
\hline 
1 & On clique sur l'exécutable de l'application & Exécutable proxy & L'application s'installe sur le client &  \\ 
\hline
2 & On vérifie que le processus associé au proxy est actif sur la machine & & Un processus associé au proxy est actif sur le client & \\
\hline
Commentaires & \multicolumn{4}{|l|}{} \\ 
\hline
\end{tabular}

\newpage

\subsection{Installation de l'autorité}

\begin{tabular}{|m{2.5cm}|m{4cm}|m{3cm}|m{3.5cm}|m{2cm}|}
\hline 
\rowcolor{Blue} Objet & \multicolumn{4}{|l|}{Installation de l'autorité} \\ 
\hline 
\rowcolor{Blue} Objectif & \multicolumn{4}{|l|}{Vérifier l'installation de l'autorité sur la machine utilisateur} \\ 
\hline 
\rowcolor{Blue} Pré-requis & \multicolumn{4}{|l|}{Autorité par défaut présente sur la machine, navigateur qui accepte les certificats MD5} \\ 
\hline 
\rowcolor{Orange} ID & Actions & Données & Résultats attendus & OK/KO \\ 
\hline 
1 & Installer l'autorité directement sur le navigateur du client & Certificat de l'autorité & Le navigateur accepte le certificat &  \\ 
\hline 
Commentaires & \multicolumn{4}{|l|}{} \\ 
\hline
\end{tabular}

\newpage

\subsection{Acceptation de l'autorité}

\begin{tabular}{|m{2.5cm}|m{4cm}|m{3cm}|m{3.5cm}|m{2cm}|}
\hline 
\rowcolor{Blue} Objet & \multicolumn{4}{|l|}{Acceptation de l'autorité} \\ 
\hline 
\rowcolor{Blue} Objectif & \multicolumn{4}{|l|}{Vérifier l'installation de l'autorité sur la machine utilisateur} \\ 
\hline 
\rowcolor{Blue} Pré-requis & \multicolumn{4}{|l|}{Pas d'autorité présente sur la machine} \\ 
\hline 
\rowcolor{Orange} ID & Actions & Données & Résultats attendus & OK/KO \\ 
\hline 
1 & Inclure le certificat dans l'acceptation des termes et conditions du proxy & Certificat de l'autorité & Le certificat de l'autorité est inclus dans l'acceptation du proxy &  \\ 
\hline
2 & Le client accepte les termes et conditions du proxy & & Le certificat de 
l'autorité va automatiquement s'installer sur le navigateur du client & 
\\
\hline 
Commentaires & \multicolumn{4}{|l|}{} \\ 
\hline
\end{tabular}

\newpage

\subsection{Transfert des paquets HTTP}

\begin{tabular}{|m{2.5cm}|m{4cm}|m{3cm}|m{3.5cm}|m{2cm}|}
\hline 
\rowcolor{Blue} Objet & \multicolumn{4}{|l|}{Transfert des paquets HTTP en clair} \\ 
\hline 
\rowcolor{Blue} Objectif & \multicolumn{4}{|l|}{Assurer le transfert des paquets HTTP du client vers le serveur} \\ 
\hline 
\rowcolor{Blue} Pré-requis & \multicolumn{4}{|l|}{Proxy configuré pour le mode HTTP} \\ 
\hline 
\rowcolor{Orange} ID & Actions & Données & Résultats attendus & OK/KO \\ 
\hline 
1 & Le client envoie une requête HTTP sur un serveur &  & Le proxy reçoit la requête HTTP &  \\ 
\hline
2 & Le proxy envoie la requête HTTP vers le serveur & & Le serveur reçoit la requête HTTP & \\
\hline
3 & Le serveur envoie la réponse au proxy & & Le proxy 
reçoit la réponse du serveur & \\
\hline 
4 & Le proxy envoie la réponse vers le client & & Le client reçoit la réponse & 
\\
\hline
Commentaires & \multicolumn{4}{|l|}{} \\ 
\hline
\end{tabular}

\newpage

\subsection{Etablissement d'une session privée}

\begin{tabular}{|m{2.5cm}|m{4cm}|m{3cm}|m{3.5cm}|m{2cm}|}
\hline 
\rowcolor{Blue} Objet & \multicolumn{4}{|l|}{Création des sessions privées entre client-proxy-serveur} \\ 
\hline 
\rowcolor{Blue} Objectif & \multicolumn{4}{|l|}{Détailler les étapes d'ouverture des sessions privées} \\ 
\hline 
\rowcolor{Blue} Pré-requis & \multicolumn{4}{|l|}{Deux connexions SSL/TLS ouvertes, certificat de l'autorité installé sur le client} \\ 
\rowcolor{Blue} & \multicolumn{4}{|l|}{et proxy fonctionnel} \\
\hline
\rowcolor{Orange} ID & Actions & Données & Résultats attendus & OK/KO \\ 
\hline 
1 & Le client envoie une requête HTTPS & URL du serveur & Le proxy reçoit la requête HTTPS &  \\ 
\hline
2 & Le proxy envoie la requête HTTPS vers le serveur & & Le serveur reçoit la requête HTTPS & \\
\hline
3 & Le serveur envoie son certificat au proxy & Certificat serveur & Le proxy reçoit le certificat & \\
\hline 
4 & Le proxy génère son certificat & Certificat proxy & Le proxy envoie son certificat au client & \\
\hline
5 & Le client reçoit le certificat du proxy & Certificat proxy & Le client vérifie la 
validité du certificat & \\
\hline
6 & Le client accepte le certificat & & Le client envoie une clé de session au 
proxy & \\
\hline
7 & Le proxy reçoit la clé de session & Clé session client & Le proxy génère sa clé de session & \\
\hline
8 & Le proxy envoie sa clé de session au serveur & Clé session proxy & Le serveur 
reçoit la clé de session & \\
\hline
9 & Session ouverte entre le serveur et le proxy & & Le serveur et le proxy 
peuvent échanger des messages chiffrés & \\
\hline
10 & Session ouverte entre le proxy et le client & & Le proxy et le client 
peuvent échanger des messages chiffrés & \\
\hline
Commentaires & \multicolumn{4}{|l|}{} \\ 
\hline
\end{tabular}

\newpage


\subsection{Transchiffrement des échanges HTTPS}

\begin{tabular}{|m{2.5cm}|m{4cm}|m{3cm}|m{3.5cm}|m{2cm}|}
\hline 
\rowcolor{Blue} Objet & \multicolumn{4}{|l|}{Transfchiffrement échanges HTTPS} \\ 
\hline 
\rowcolor{Blue} Objectif & \multicolumn{4}{|l|}{Transchiffrement sur les connexions HTTPS de manière transparente} \\
\rowcolor{Blue} & \multicolumn{4}{|l|}{pour l'utilisateur} \\ 
\hline 
\rowcolor{Blue} Pré-requis & \multicolumn{4}{|l|}{Session privée établie, Certificat généré, deux connexions SSL/TLS ouvertes,} \\ 
\rowcolor{Blue} & \multicolumn{4}{|l|}{certificat de l'autorité installé sur le client et proxy fonctionnel} 
\\
\hline 
\rowcolor{Orange} ID & Actions & Données & Résultats attendus & OK/KO \\ 
\hline 
1 & Le client envoie une requête HTTPS au proxy & URL du serveur & Le proxy reçoit la requête HTTPS &  \\ 
\hline
2 & Le proxy déchiffre la requête HTTPS & & Le proxy log la requête en clair & \\
\hline
3 & Le proxy chiffre la requête & Clé session proxy-serveur & La requête est 
correctement chiffrée & \\
\hline
4 & Le proxy envoie la requête chiffrée au serveur & & Le serveur reçoit la requête & \\
\hline
5 & Le serveur envoie une réponse au proxy & & Le proxy reçoit la 
réponse & \\
\hline
6 & Le proxy déchiffre la réponse du serveur & & Le proxy log la réponse au 
serveur en clair & \\
\hline
7 & Le proxy chiffre la réponse & Clé session client-proxy & La réponse est 
correctement chiffrée & \\
\hline
8 & Le proxy envoie la réponse chiffrée au client & & Le client reçoit la réponse 
& \\
\hline
9 & Le client déchiffre la réponse & & Le client peut lire et traiter le réponse 
& \\
\hline
Commentaires & \multicolumn{4}{|l|}{} \\ 
\hline
\end{tabular}

\newpage

\end{document}